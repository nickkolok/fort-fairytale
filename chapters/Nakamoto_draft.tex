///\{\{ Драфт сюжета


После посадки Экзюпери на <<Голландец>> Форт начал активно развивать воздухоплавание на основе драконов.
Сёдла, ремни наподобие автомобильных, для перелётов на большой высоте - сёдла в виде комбинезонов из драконьей же кожи
+ соответствующие заклинания для согревания. Альтернативно --- заклинание-<<кабина>>.

Катрин и шархи Джеронимо затаскиваю Сатоши на дракона, крепят его и себя ремнями и взлетают.
Шархи вырубается, страхуя беглеца в Лабиринте.

В Форте принимают решение: лететь от Алеутских островов, где расположен Нью-Нокс, на Камчатку,
там садиться и разбираться, что произошло. Лететь примерно 36 часов (скорость дракона где-то 40 км/ч).
Партизанская неуловимость, наложенная Катрин, вкупе со способностью драконов, подобно хамелеонам,
менять цвет кожи и с тем фактом, что металла на драконе немного и потому для радаров он невидим,
позволяют им пробираться незамеченными на высоте порядка километра.

К восходу солнца Сатоши начинает приходить в себя и задавать вопросы.

--- Где мы?

--- На драконе.

--- Куда вы меня везёте?

--- Туда.

Катрин отказывается назвать время, хотя часы у неё есть, ссылаясь на часовые пояса.
Сатоши планировал использовать время, чтобы определить, не опережают ли они вращение Земли
(за время пребывания в тюрьме его психика сильно накренилась).

Дракон не может преодолеть такое расстояние за один перелёт. Форт через портал присылает дракона на смену.

Пересадка между драконами осуществляется следующим образом.
Новый дракон зависает над старым, наездник нового дракона сбрасывает тросы с сидений вниз.
Пассажиры на старом драконе крепятся к этим тросам.
Новый дракон уходит вниз и вбок, пассажиры скользят по тросам на нового дракона.
Джеронимо <<вырубает>> Сатоши и себя, Катрин способна потреять сознание самостоятельно (уже научили) во избежание ненужного стресса.
Тросы надежны, ибо наложена нерушимость, а там уж топология накосячить не даст.

Сатоши делает предположение, что второй дракон переместился в пространстве много быстрее первого,
следовательно, Катрин и компании доступны перемещения сквозь пространство, но почему-то его,
Сатоши, таким образом не транспортируют. Объяснение он видит либо в желании произвести впечатление
(что отнюдь не добавляет ему дружелюбия), либо в невозможности транспортировать произвольное живое существо.

Кроме того, японец держит в уме версию, что это всё инсценировка с использованием соответствующих технологий
(об уровне развития которых он в основном может только догадываться --- около десяти лет в тюрьме).
Последнюю версию он высказывает Катрин и предлагает ей опровергнуть.

После посадки на Камчатке Катрин вторично объявляет, что Сатоши свободен.
Японец воспринимает предложение безоружным идти по тайге куда глаза глядят как издевательство и просит оружие.
Катрин отказывает: поставки ещё не налажены, каждый ствол на счету, но она об этом не говорит.

Сатоши заявляет, что раз он свободен, то намерен следовать за Катрин. Катрин, по совету Хранительницы, отвечает,
что они пока остаются здесь, и раскатывает стандартное антимоскитное заклинание.
Наблюдая за переговорами Катрин по зеркалу, стилизованному под смартфон, во время полёта, японец замечает,
что понимает Катрин, но не понимает её собеседника, и запоминает этот факт.

Еду и всё необходимое Катрин приносит через портал, открываемый в сотне метров по утрам, пока японец отсыпается.

Отсутствие комаров, непонятно откуда берущаяся провизия и непонятная речь, доносящаяся из смартфона,
утверждает его в мысли об инсценировке.

Хранительница успокаивает Катрин: тушение лесных пожаров продолжается (как раз лето),
Линдариэль на пару с Громостонко разобрались в системе оповещений МЧС, инфраструктурная дыра в которую проложена Катрин,
а значит, <<лишняя>> энергия пока есть, и кому, как не Катрин, этими излишками по праву распоряжаться.
(Напомним, что стратегическим источником считать внешний мир нельзя, но на разовую операцию --- а почему бы нет.)


Джеронимо засыпает синхронно с Сатоши, прикрывая его в Снах --- Верхнем Лабиринте --- на случай рецидива болезни
(<<Подсак Посейдона>> поддерживать слишком затратно, расстояние).
Во время одного из дежурств Сатоши находит американский шархи-сноходец и пытается на него воздействовать,
но замечает Джеронимо и сбегает.

Сатоши рассказывает, что американцы режиссировали ему сны (и таким образом выведали некоторую информацию, пока он не понял),
а затем и пытали в снах с целью склонить к сотрудничеству. Поэтому же он воспринял приглашение в Форт,
также переданное через сны, как инсценировку. Да и сейчас, говорит он, гипотеза непротиворечива.
Кроме того, он добавляет, что шархи-американца он узнал.

Джеронимо объявляет срочную побудку, поднимая Катрин, и сообщает в Форт о том, что государствам внешнего мира
(как минимум, одному) служат шархи.
Хранительница собирает Совет. Громостонко (замещающий Катрин как главу Школы Янтарной Мудрости)
предлагает объявить общую побудку, включая детей, осадное положение и т. д.
Глава школы сноходцев объясняет, что ничего страшного не произошло и в <<лабиринтальный карантин>>
требуется только поместить Джеронимо и Сатоши, следя, чтобы они не контактировали в Лабиринте и снах с другими адептами.

Совет объявляет поиск добровольца, готового прикрывать Сатоши и Джеронимо в снах (один боец --- ненадёжно).
Супруга Джеронимо --- не сноходец, её кандидатура, наиболее логичная --- отпадает.


Бой за Нью-Нокс активно обсуждается, в том числе среди молодёжи.
У Сатоши образуется даже своеобразный фан-клуб (девочки-подростки везде одинаковы).
Одна из членов этого фан-клуба, Рипануэль, вызывается добровольцем и переправляется через портал.

Катрин представляет Рипануэль и Сатоши друг другу и поясняет, что теперь <<вы будете спать вместе>>.
Сатоши гневно заявляется, что не собирается трахаться с малолеткой. Рипануэль смущается.
Сатоши в ярости спрашивает, а готова ли очередная кукла кей-джи-би сделать ему минет.
Рипануэль спрашивает Катрин, что это такое. Услышав объяснение, Рипануэль восклицает <<Ну вы там даёте!>>,
Сатоши запоминает это как странность.
Рипануэль просит у Катрин разрешения погулять вокруг места стоянки. Катрин увязывается следом.
На очередном круге Сатоши слышит, как Рипануэль восклицает: <<Он не будет меня трахать, потому что я страшная, да?!>>.
Катрин с трудом успокаивает её.



Шархи-американец сообщает руководству о столкновении.
Рабочая гипотеза с учётом инопланетного контекста --- инопланетяне нашли и завербовали сноходца.


/*
Здесь напрашивается идея искать наземную базу.
Нет, конечно, шархи может мирно дремать в постели в своей квартире и получать свой унобтаниум,
бессметрие или чем там с ним НЛО расплачивается, но почему бы на всякий случай не поискать?
*/


Где-то раньше: 1945-й. Японская пара.
Он заявляет, что боги не дают им детей, потому что хотят послать знак, что он должен стать камикадзе.
На следующее утро он берёт самураский клинок и выходит (вылетает?) на бой.
Через несколько дней она узнаёт, что наконец беременна.



В Форте изучают трофеи, вынесенные из Нью-Нокса --- золото, оружие, какие-то документы и самурайский меч в футляре.
Хранительница отправляет его Катрин с просьбой показать Сатоши, так как маги чувствуют магию, исходящую от клинка.


При виде меча Сатоши приходит в замешательство и спрашивает, откуда этот клинок. Катрин отвечает, что взят в качестве трофея.
Сатоши говорит, что это --- родовой меч его деда (прадеда?), он чувствует,
что клинок настоящий и что американцы уже пытались его этим клинком <<купить>>, но он ответил <<Не из твоих рук>>.
Вероятно, американцы сочли, что чин предлагавшего меч был недостаточно высок, позвали начальника тюрьмы. Тот же ответ.
Клинок положили дожидаться приезда какого-нибудь начальства, посла Японии и т. д.

Но отказ принять меч значил совсем не это: Сатоши не может принять меч из рук врага.
А вот если союзник разгромил врага и возвращает ему меч предков --- тогда пожалуйста,
и он даже присягнёт тому, что вручит клинок. Но осталось самая малость ---
убедиться, что это не инсценировка, и для этого Катрин должна коснуться меча и сказать,
что клинок действительно взят как трофей третьей силой, враждебной Америке.
Сатоши говорит, что поймёт, правда это или нет. Катрин проделывает требуемое.
(Джеронимо в это время временно отбыл в Форт).
Сразу после того, как она убирает руку, Рипануэль касается клинка и признаётся, что влюблена в японца и хочет его.

Сатоши удивляется и просит у неё прощения. Катрин спрашивает, как он узнал, что обе сказали правду.

--- Очень просто, --- отвечает японец, --- в наказание за ложь клинок выпивает душу.

Катрин повторяет приглашение в Форт и просит Сатоши больше таких фокусов не вытворять.
Японец отвечает, что отныне её слово для него приказ (так как она вернула ему меч).


Портал, локальный хэппи-энд.


///\}\} Драфт сюжета

