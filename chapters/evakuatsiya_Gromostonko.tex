Решение районного Трибунала по противодействию терроризму.

Именем Закона и Безопасности Суд,
проведя заседание в закрытом режиме в соответствии со ст. 13 КФЗ-666 <<О цифровом суверенитете>>, постановил:

а) Статью <<Квазилинейные неинъективные операторы в пространствах Орлича>>,
опубликованную в периодическом издании <<Успехи математических наук>> в 2021 году за авторством Громостонко И.А.,
признать содержащей сведения,
которые могут быть использованы в целях построения информационно-телекоммуникационных сетей вне контроля
Правоохранительных Органов, и внести в Конфедеральный перечень экстремистских материалов.

б) Запретить распространение периодического издания <<Успехи математических наук>>
за 2021 год посредством информационно-телекомму\-никаци\-онных сетей, включая, но не ограничиваясь, сеть <<Интернет>>.
Изъять материальные и цифровые экземпляры данного издания из частного и библиотечного пользования,
в том числе с применением автоматизированных средств изъятия информации. Материальные экземпляры уничтожить.

в) Громостонко И.А. признать виновным в неумышленном пособничестве терроризму и назначить ему наказание в виде:

1) Лишения свободы на срок 3 года условно с испытательным сроком 3 года.

2) Принудительных работ по специальности сроком 18 000 часов.

3) Лишения права ведения педагогической деятельности пожизненно.

4) Штрафа в виде 1000 (одной тысячи) минимальных размеров оплаты труда.

г) Громостонко И.А. явиться в трудовой комиссариат по месту жительства
для ознакомления с графиком работ в течении суток с момента получения уведомления о решении Суда.

\emptypar

Доцент скомкал листок оповещения (неуважение к суду --- до трёх лет, но квартира не ипотечная, камеры не установлены),
в глазах его потемнело. Всё. Больше никакого семинара. Никаких лекций.
Прощайте, научная работа и мечта жениться на хорошей студентке. Здравствуй, шаражка.
Денег на оплату штрафа у бедного преподавателя нет --- придётся заложить квартиру, а значит, и жить онлайн,
под пристальными взглядами. Да ещё и работать на тех, кого всю университетскую жизнь ненавидел он сам и большинство его коллег.
На тех, кто год за годом сжимает удавку на горле науки --- прикрываясь то <<безопасностью наших детей>>,
то <<авторским правом>>, то <<оптимизацией>>, то <<новыми образовательными стандартами>>.

Остаётся единственное верное решение. Окинуть взглядом знакомый ковёр над кроватью.
Вспомнить, как в детстве, засыпая, разглядывал на нём узоры.
Мысленно попрощаться со стареньким дребезжащим холодильником и с облупленной настольной лампой.
Встать на стул и приладить галстук к багете.

И услышать женский голос:

--- Вы ошиблись, Иван Алексеевич. Решение не единственно.

\emptypar

Пока ошарашенный доцент молчал, пытаясь сообразить, чей это голос и откуда он взялся в его квартире,
невидимая собеседница продолжала:

--- Извините, можно войти?

Память услужливо зашуршала страницами...

\begin{achronic}

...Доцент Громостонко очень не любил, когда студент приходит не вовремя. Даже на семинар --- и все это знали.
И в этот раз он уже протёр доску и собрался огласить план занятия, когда дверь приоткрылась.

--- Извините, можно войти?

На пороге стояла Катя Жужелицына --- прилежная отличница со второго курса, выделявшаяся стопроцентной посещаемостью.
И преподаватель, ворча, сделал исключение.

--- Спасибо, --- проговорила девушка, --- я больше не опоздаю.

\end{achronic}

--- Жуж... Катрин... Да что ж это такое? Пардон, всё никак не привыкну к Вашему новому имени... Но как?!
Вы специально следили за мной?

--- Конечно, нет, Иван Алексеевич! --- девушка тряхнула головой, отбрасывая за плечи распущенные волосы,
--- Я следила отнюдь не за Вами, а за базой подобных решений.

--- Час от часу не легче... То есть они что, до сих пор не позакрывали дыры двадцатилетней давности или,
хуже того, наделали новых, так что каждый может это отслеживать?

--- Каждый, у кого есть физический доступ во внутреннюю сеть соответствующего ведомства.

--- Конечно, я люблю загадки, но - такие, какие могу разгадать... Как, скажите мне, как?!

--- Да так же, как Вас эвакуировали. Провод в микропортал пять на пять миллиметров, а пароли и зеркалом подсмотреть можно.
Потом выбрала тех, кого в случае чего нечем будет прижать --- ни детей, ни родителей --- и кто занимается нужными нам задачами.

--- Друг мой, --- заметив укоризненный взгляд учёного, заговорила хранительница, ---
в этом и заключается одна из страшнейших драм Форта: мы можем помочь любому, но не можем помочь всем.

--- Пока не можем, --- хитро улыбнулась Катрин и склонила голову на бок.

\emptypar


\emptypar

