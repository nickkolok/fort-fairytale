 - Разойдитесь, братия!
Толпа зашумела, задвигалась и наконец расступилась, пропуская рыцаря в белом плаще с крестом. Где-то раздались возгласы:
 - Вот ужо сейчас его!
 - А у ентого горе-героя кол-то осиновый есть?
 - Дурак ты, Джон! Серебром эту харю, серебром.
Рыцарь подошёл к связанному человеку. Мужчина, средних лет на вид, слегка помятый и связанный селянами, но ещё в сознании.
 - \_Не бойся,\_ - скрипнула латынь.
И, обернувшись к не услывшавшей последней фразы толпе, уже с расчётом на англичан спросил:
 - Кто это? Какого рода?
 - Да пришлый он! - выкрикнул кто-то. - Вот как поселился здесь надысь, так вот и овцы стали попадать, а третьего дня цельную корову сожрал. Уууу, гад!
 - Резать его!
 - Добейте его, рыцарь!
 - Спокойно, дети мои, спокойно. Я здесь, чтобы именем Мальтийского ордена не дать вам совершить роковую ошибку. Вы и вправду хотите осквернить его чёрной кровью родную землю?
Толпа задумалась. На простых крестьянских лицах отражалась работа мысли (на некоторых - потеснённая желанием поскорее всё закончить и отметить это дело). Наконец кто-то выкрикнул:
 - Тады сжечь его, поганца!
 - И вдохнуть прах сожжённой нежити? - ответил рыцарь.
Толпа снова загудела.
 - Я наделён даром упокаивать подобную нечисть, - продолжал рыцарь, - но мне нужна кружка вина и кружка воды.
Примерно половина собравшихся ринулась по домам или к ближайшему колодцы; через несколько минут в руках у крестоносца наконец оказались требуемые предметы; когда жидкость коснулась тела оборотня, серебристое пламя поднялось от его тела вверх... И больше он не дёрнулся.
Мальтиец наклонился, забросил неподвижное тело себе на плечо:
 - Благодарю за усердное содействие, братья. Мы погрузим его на корабль и утопим в дальних водах...
Опять нестройный гул ответных благодарностей. Рыцарь повернулся и зашагал прочь.


Пленник, освобождённый от парализующего заклинания (с куда меньшими спецэффектами, чем те, которыми сопровождалось наложение оного), тяжело отдышался. Из под грязных волос полоснул затравленный взгляд:
 - Ты человек. Не из наших. Что тебе от меня надо?
 - Начнём с того, - "рыцарь" снял шлем и уселся на траву рядом, - что не совсем человек, а четверть-эльф. И предлагаю я тебе вот что...
Спасённый слушал внимательно, исподлобья глядя на рассказчика. Думал, жуя травинку. Но упорно молчал.
 - Так ты согласен?
Оборотень вздохнул и покачал головой.
 - Нет, пока не пойму, что вам от меня надо. Хоть ты меня и спас, но я тебе не верю.
Теперь настали очередь вздохнуть рыцарю.
 - Почему?
 - Ты не обманешь мой нюх. Ты не говоришь мне всего. Ты не объясняешь, зачем вы предлагаете сменять голодную свободу на сытую скуку. Зачем я вам нужен?
Ага, манускрипты не врут. У оборотней действительно есть способность определять не только ложь, но и полуправду. Он хотя бы не телепат?
 - Таких, как ты, осталось мало... - осторожно начал парламентёр.
 - И какое вам до этого дело?!
 - ... и нам бы хотелось обеспечить вам возможность размножаться...
 - Темнишь, человек! - оборотень начал приподниматься. - Зачем я тебе?
 - Нам нужны бойцы. Практически неуязвимые бойцы, которых нужно прикрыть только от серебра.
 - Я. Не. Буду. Ничьим. Служакой.
Лжерыцарь убеждать умел плохо. Особенно плохо он умел убеждать стихийных полутелепатов...
 - Хорошо. А как насчёт перепихнуться?
 - С тобой, что ли?
 - Да нет, мы тебе хорошенькую девушку подгоним. Кого любишь, блондинок, брюнеток, рыжих?
 - Мёртвых, - коротко отрезал вервольф. - Не хочу, чтобы у меня были дети. Никому не хочу такой судьбы.



- И примерно вот так было несколько раз, - вздохнула Хранительница, - кто-то отказывался, к кому-то просто не успевали. А большинство просто не нашли. Направленного сигнала-то о помощи не было, так, засекали по стихийным всплескам... С оборотнями не церемонились. Их не брали в плен. Их убивали на месте...
 - А в чём прикол, зачем они так нужны? - спросила Катрин, - Нет, конечно, ясно, что такие способности - это круто и всё такое, но...
 - Дитя моё, даже уникальных способностей хватило бы с лихвой, чтобы обеспечить им место в нашем заповеднике. Кстати, у меня будет к тебе разговор по поводу твоей вторичной эгидали... Но это потом. Но весь смысл в том, что щиты оборотня не получают магию от Форта. Он сам себе источник магии. Он сам обеспечивает себе регенерацию, если вдруг его поранят, что весьма и весьма маловероятно; а если оборотня воспитывать с младенчества, то можно создать медальон, позволяющий менять облик по желанию... По крайней мере, так написано в манускриптах. Вообще, знаешь, чуднО всё это... Мне порою кажется, что их магия - такая же, как магия Форта, но не прикреплённая к конкретному месту... Мы думали, их вообще не осталось. Перебили всех. Мало кто из мужчин ни разу в жизни не попадал в лунный луч в присутствии кого-то ещё... Но кровь перешла по женской линии... Дальше ты знаешь.

