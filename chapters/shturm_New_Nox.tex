

--- Мы делать выстрел, если вы не уйти! --- солдат-американец явно чувствует своё превосходство, картинно прицеливаясь из автомата.

--- Мы <<не уйти>>, --- плеснула ненависть из серых глаз сквозь лобовое стекло <<копейки>>,
--- а спустишь курок --- пожалеешь.

<<Какие же они смешные, эти русские романтики... Наверняка водку пила>>, ---
подумал рядовой Райан, подошёл почти в упор, метра на два, и, прицелившись в голову, выдал короткую очередь.

...Одна за другой пули срикошетили от магического щита. Девушка высунула руку с пистолетом из окошка и наугад выстрелила.
Односторонний щит с негромким чавканьем пропустил пули; вреда солдату, однако, даже те две, что попали в цель, не причинили.

\emptypar

--- Теперь наш ход, --- раздалось откуда-то сверху.

У самого берега острова-тюрьмы стоял Форт.
Волны, недоумевающие, почему простор для их бега так беспардонно сократили, яросто бились о гранитные стены.
На центральной башне, положив руки на круглый кристалл, стояла женщина в чёрной мантии, развевающейся на ветру.

Райан, судорожно зажав курок, начал было перенаправлять дуло вверх, но не успел. Старик на одной из башен опередил его.
Как только первые пули коснулись серого камня, из посоха ударил тонкий зелёный луч.
Звёздочка перед глазами, выжигающая мозг. Темнота. Тоннель.

Катрин отточенным движением отпустила сцепление и рванула к трупу. 
абросить на заднее сиденье, как бы противно и страшно ни было...

<<Копейка>> --- насколько смогла резко --- развернулась и понеслась к опускающемуся мосту. Ей вслед уже летели пули и гранаты, но девушка-водитель лишь ухмылялась --- теперь, под щитами, они ей были не страшны.

Хранительница улыбнулась, с некоторым усилием вызывая цепочку образов. Ещё один скачок.

Командир боевой группы не видел, как на кончике посоха некроманта набухает и дрожит чёрно-бардовый шар.

Перстень на пальце эльфийки вспыхнул. Магичка вскинула руку с кольцом к глазам, прищурилась, целясь,
и прикоснулась к камню другой рукой.
Взмах --- и <<кинжал Калиостро>> срывается с ладони,
неестественно ровная линия проходит по бетонной стене самой неприступной из тюрем оплота демократии.

<<Поцелуй пустоты>> семигранной пирамидой прошёлся по небольшому плацу, слизывая военных.
Подразделение, оставшееся под защитой укреплений, спешно перегруппировывалось,
пытаясь вести ответный огонь и не понимая, почему это интеллектуальные самонаводящиеся ракеты не долетают.

Толстенная метровая стена, словно ломтик хлеба, рухнула в воду, обнажая камеры. Падая, она погребла под собой патрульный крейсер...

\emptypar

--- Вот он.

Ещё один скачок портала. Угол моста аккуратно ложится на холодный пол мешка полтора на полтора метра.
Пожилой азиат --- единственный узник Нью-Нокса, удерживаемый, по издёвке судьбы и правительства, рядом с мировыми запасами золота,
которое так ненавидел всю жизнь, --- сидит, ошалело прижавшись к стене.
Слышно, как в коридоре грохочут сапоги надзирателей, спешаших уничтожить пленника.

Катрин вскидывает руку, импульс срывается с распрямляющихся пальцев ---
и тяжеленная, укреплённая всеми мыслимыми и немыслимыми способами дверь вылетает в коридор, впечатывая в стену отряд.
Ей остаётся только сойти с моста на пол и подойти к старику.

--- Сенсей Накамото, Вы свободны.

\emptypar

--- Свобода, кей-джи-би-тян? Ты всерьёз полагаешь, что я променяю одну тюрьму на другую?

Катрин покачнулась. Красивый сценарий, какой её и не снился лет пять назад,
когда она в этой самой одежде корячилась на университетской физре, летел к чертям.

--- Сатоши-сан, но Вы же...

--- Я обещал, что не покину пределы Форта, если ступлю на его землю. Но я не обещал, что куда-либо пойду.

--- И да, между прочим, я не из КГБ. КГБ давно нет... И не из КФМБ.

--- Ох-ох-ох, постарел я, --- тощее лицо японца исказила глумливая гримаса, --- значит, это всё просто красивая картинка?
С чего вы вообще взяли, что я с вами пойду?

С трудом подавив желание ляпнуть <<Но другие же шли>>, Катрин, замолкнув на пару секунд, неуверенно протянула:

--- Например, Вы сможете заниматься любимым делом...

--- ...на благо моей новой родины? --- усмехнулся Накамото, --- Это мы уже проходили. Вы неоригинальны.

--- Так почему вы сделали вид, что согласны на эвакуацию, чёрт возьми? ---
Катрин начинала уже злиться, но годами выработанная сдержанность заставляла её не повышать голос.

--- Я хочу умереть. Всё, хватит. Сатоши кончился. Для интернета я умер восемь... или девять, со счёту я сбился...
лет назад... В 2015-м, после падения TrueCrypt...

--- Сейчас 2024. И если Вы не покинете рушашийся Нью-Нокс, то, боюсь, Ваше желание исполнится.

--- Вот и славно. Ты провалила задание, кей-джи-би-тян.

\emptypar

Катрин склонила голову. Как обычно бывает, мозг запоздало подбрасывал аргументы. Секунды медленно стекали с каменного пола.

Где-то там, возможно, уже нажата красная кнопка, и стая <<Томагавков>> рыщет в таком непривычном анизоптропном пространстве,
целясь в стены Форта. Или хуже --- в щель между стенами и краем портала.
Девушка тряхнула головой, зачем-то одёрнула чёрную блузку.

--- Вот Вы говорите о свободе... За девять лет мир изменился.
Фермер Джон, который берёт кредиты, чтобы купить трактор, рассаду и топливо,
а потом выплачивает и снова берёт, берёт и выплачивает --- он свободен?
Какой-нибудь Хуан, который горбатится на наркобарона на кокаиновой фабрике просто потому,
что другой работы в стране нет, а по границе --- колючая проволока --- может быть, свободен он?
Или, может быть, свобода --- это когда три десятка школьниц голышом стоят перед гинекологом-инспектором ради
<<демографической стабильности>>?
Может быть, наконец, единственный островок свободы --- это славное государство Северная Корея,
где портрет вождя надо протирать зубной щёткой?
Или Ваша родная Япония, где воздух для дыхания покупают по счётчику, как воду?
А-а, знаю. Свободны те, у кого много денег. Те, кто может всё купить.
И, конечно, их совсем не тяготят чемоданы компромата, собранные такими же зубастыми лжедрузьями. Уж про политиков молчу...

--- К чему эта пропаганда, кей-джи-би-тян? --- недовольно прервал её узник, --- чего вы от меня хотите?

--- Чтобы Вы головой подумали, в конце-то концов! Вы же математик! Ма-те-ма-тик!
Во-первых, если бы мы могли Вас взять силой --- стала бы я тут разводить политпросвет?
Принципиально важно, чтобы Вы сами выбрали этот путь. Во-вторых, о любимом деле...
А просчитайте, что будет с курсом доллара и со всей золотовалютной системой. Посмотрите вооон туда...
На половину эсминца. У наших бойцов ещё есть энергия, чтобы обеспечивать нашу с вами болтовню, но не так уж много.
Ребят, которые сейчас опустошают сам золотой склад, я при всём желании показать не смогу --- зеркала нет.
Просчитайте... у Вас есть шанс дожить до падения фиатки. Вот оно, близко.
Пара часов --- и новость разлетится по миру, биржи забьются в агонии... Вот она --- победа криптовалют.
Вы сделали свою работу, дальше мы справимся сами. А Вам предлагаем лечение, мягкую кровать, горячую ванну и регулярную еду.
Даю слово линуксоида, что заставлять Вас заниматься наукой никто не будет. Максимум --- докучать расспросами.
Словесными, --- Катрин презрительно махнула головой в сторону выбитой двери, --- а не как некоторые.
Да, и в пределах наших небольших границ Вы будете свободны. Свободнее, по крайней мере, чем Ассанж в посольстве.

Японец молчал, прикрыв глаза. Катрин остановилась, вдохнула. Выдохнула и снова вдохнула.

--- Да хватит уже упрямиться, Сатоши!

В ответ тишина. Где-то там, под рёбрами, начало зарождаться, скребясь когтями, смутное, но страшное подозрение.
На с трудом слушающихся ногах Катрин подошла к японцу и нащупала пульс.

Вернее, попыталась нащупать.



\commentize{
/*

--- В левое запястье погибшего, --- голос медикуса-патологоанатома звучал абсолютно спокойно, --- был вшит металлический предмет неизученной структуры, оснащённый приспособлением для впрыска быстродействующего яда. На момент поступления в лазарет этот человек был уже мёртв, а в таком случае мы бессильны.

--- Сссуки... --- процедила Катрин, --- канареечный сигнал. Как же я не догадалась!

--- Ты что-то поняла, дитя моё? Во внешнем мире изобрели новый способ заочного умерщвления?

--- Примерно. В руку Сатоши был вшит --- вероятно, когда он был без сознания, но теперь не узнаешь, --- специальный чип. Металлический. Этот чип бездействовал, пока принимает радиосигнал от систем тюрьмы. Когда мы включили трофейные "Грабли", радиосигналы перестали приходить. Чип, возможно, подождав для верности несколько минут, активировал капсулу с ядом.

--- Что ж, дитя моё, полагаю, нам следует смотреть на вопрос философски. В то, что за ним открылся малый портал, сей учёный муж бы не поверил; а предугадать столь подлый метод умерщвления... Полагаю, орудие бы впрыснуло яд и в случае, если бы его попытались извлечь?

--- Не знаю, --- пожала плечами Катрин, --- скорее всего. Надо будет пошукать по закрытым сетям.

*/
}

\emptypar


<<Подсак Посейдона>> взрезал Астрал, перебивая занесённую косу у самой рукояти --- и вернулся не пустым.

Девушка тяжело привалилась к каменной стене.
Заклинание словно давило на плечи, оттягивая их назад и вниз,
как будто тяжесть борьбы с чужой смертью была аккуратно упакована в огромный рюкзак.
Забыв про всё, что происходит вокруг, она думала лишь о том, какая она самонадеянная дура.
Во-первых, только что она чуть не завалила всю операцию, занявшись ораторством вместо того, чтобы сразу,
как предписывает веками выверенная инструкция, подстраховаться заклинанием.
Во-вторых, вся эта затея... А что, если он не согласится?
В-третьих --- мысли ускорялись, словно ржавая шестерёнка где-то под рёбрами --- даже сейчас она теряет драгоценное время...

--- Врача! Тьфу, медикуса! Беглец без сознания!!!

\emptypar

Доцент перевёл взгляд с зеркала на Хранительницу и изрёк:

--- По-моему, у неё проблемы. Они говорят об очень странных вещах.

Хранительница мысленно помянула недобрым словом ограничения языковой магии,
требовавшей непосредственного пересечения речевых контуров --- или <<переводчика>>-посредника.

--- Подробнее же, друг мой.

--- Японца я понять не могу. Катрин выражает недоумение, пытается доказать,
что она не из КГБ --- это такая инквизиция в Советском Союзе, в двадцатом веке...

--- Благодарю, я в курсе. Время.

--- А теперь пошли рассуждения о свободе. Да, он отказывается идти. Катрин обещает ему безопасность и уважение...

Проверку пульса и отчаянный обкаст оба увидели одновременно.
Громостонко хотел что-то сказать, но увидел, как с Площади Приветствий быстрым шагом выходит пара дежурных --- врач и шархи.

\emptypar

--- В левое запястье пациента, --- голос медикуса звучит абсолютно спокойно,
--- вшит металлический предмет неизученной структуры, оснащённый приспособлением для впрыска быстродействующего яда.
Детоксицирующее заклинание я наложил, но до того, как пациент придёт в себя, пройдёт от нескольких часов до нескольких дней.

Форт не любил спешки. Зачем ускорять методику, использовавшуюся десяток раз от силы?
Добравшемуся до этих стен спешить уже некуда...

Медикус достал связное мини-зеркало, настраивая его на центральную башню.

--- Катрин, Хранительница просит Вас не переживать и говорит, что это в любом случае её вина,
поскольку она дала добро на операцию. Можете объяснить, что произошло?

--- Она и не собирался идти с нами, --- тихо вздохнула девушка, --- он сказал, что хочет умереть.

Бронзовый скрип латыни.

<<Надо же, --- подумалось вдруг Катрин, --- никогда бы не подумала, что зеркало бывает языковым барьером...>>

\emptypar

Серый дракон завис в воздухе, положив шею на мост. Наездник приглашающе махнул рукой:

--- Поднимаемся, и побойчее.


