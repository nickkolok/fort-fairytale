Часы в Зале Совета пробили октаву до зенита.
Огонь в камине, вдохвовлённый отсутствием окон --- а если быть точным, то зеркал, их заменяющих, ---
радостно сыпал бликами на лица присутствующих.

Ультрамэтра Нортога и субультрамэтрэссы Бореиды.

Ультрамедикуса Клементины и субультрамедикуса Луизы.

Ультрашархи Шарля и субультрашархи Жаннетты.

Драконария Томаса.

Мечника Юджина.

Смотрителя библиотеки Джованни.

Хирдового Рудгерра XXXVI.

%TODO: я кого-то наверняка забыл. Вписать, чтобы было нечётное число (без учёта Катрин).

Первого наследника Жана дель Кастелладо.

Хранительницы Мари дель Кастелладо.


И Катрин, неуютно озирающейся под взглядами собравшихся и прижимающей к груди ноутбук.


--- Итак, друзья мои, рада вам сообщить, что вновь прибывшая беглянка делает значительные успехи, ---
начала Хранительница.

--- Вы считаете, что это повод собрать Совет посреди лунной квадры, оторвав нас от дел? --- заскрипел Нортог.

Хранительница молчала, едва заметно мягко улыбаясь.

--- И список причин, по которым такое может состояться, --- вторил ему смотритель библиотеки,
--- весьма ограничен. Это вопрос о принятии нового беглеца...

Хранительница промолчала, чуть-чуть покачав головой.

--- ... смерть Хранителя, рождение или смерть первого наследника...

Мари дель Кастелладо насмешливо посмотрела на Джованни. Мол, а сами не видите?

--- ... вопросы войны и мира, --- уже увереннее полуспросил смотритель.

Женщина безмолвстовала.

--- ... угроза безопасности Форта ...

Хранительница вздохнула.

--- ... и, наконец, выход магии из-под контроля, требующий немедленного всеобщего вмешательства, ---
мрачно пригвоздил Норгот. --- Кажется, я понимаю, какого рода успехи делает девушка, --- и кивнул в сторону Катрин.
--- Дело молодое, опыта нет, руки горячие...

--- Вы кое-что забыли, друг мой, --- тихо сказала Мари и добавила, --- шесть на семь футов.

\emptypar

Секунды три висела тишина.
Нортог вышел из замешательства первым и приподнялся, опираясь на стол.

--- Я хочу это видеть. Немедленно.

Клементина встала и кивнула.

--- Друзья мои, ваше любопытство будет удовлетворено. Но дайте же мне завершить речь!

--- Кстати, --- зашелестел смотритель, --- а при чём тут девушка?

--- Говорю же, друзья мои, вы излишне спешите. Катрин де Лада, Вам слово.

Катрин медленно приподнялась и бережно положила ноутбук на стол.

--- В общем, --- девушка чувствовала себя, как будто попала на так и не состоявшуюся защиту своего бакалаврского диплома,
--- я поняла, что заклинание есть хэш, и сбрутила его на ноутбуке.

Тишина вернулась. Нависла серыми плечами над каменным столом. Упёрлась в стены. Казалось, что растерялся даже огонь.

Но не Хранительница.

--- Дитя моё, прошу тебя... Оставь термины специалистам, --- в голосе Мари промелькнули саркастические нотки, взгляд
почти невидимо скользнул в сторону Нортога.

--- Извините, --- девушка опустила глаза. --- У меня с собой был ноутбук. Это такой стандартный артефакт.
Я заложила в него алгоритм расчёта, --- девушка, поймав укоризненный взгляд Мари, осеклась,
--- то есть объяснила на понятном артефакту языке, как открывается портал. Из этого вашего трактата...

--- Манускрипта, --- поправила её Хранительница.

--- ... манускрипта, --- эхом повторила Катрин. --- Артефакт немного подумал и выдал мне варианты заклинаний...

--- Сколько? --- кратко спросил Нортог.

--- Тысяча четыреста шестьдесят восемь, --- ответила Хранительница.

--- И все ведут к открытию порталов шесть на семь? --- подал голос Юджин.
--- Двоим моим ребятам в такой не втиснуться, но гуськом --- очень удобно.

--- Я проверила двадцать три из них в порядке утренней квоты на интуицию, ибо нет разницы, как угадывать. ---
сказала женщина, --- одно оказалось верным.

--- Так это просто Фортуна, --- не то облегчённо, не то разочарованно выдохнул Нортог,
--- девушке просто повезло.

--- Что не отменяет моих поздравлений, --- Шарль улыбнулся просто и открыто, как всегда улыбаются шархи.

Катрин исподлобья бросила на собравшихся обиженный взгляд.

--- Это не случайность.

--- Да, --- заскрипел Нортог, --- а почему тогда не все заклинания открывают портал? Артефакт мощью не вышел?

--- В алгоритме хэширования... то есть в манускрипте, ошибка. Не хватает условий.

--- Вы много на себя берёте, --- прозвенел голос Луизы, --- критиковать текст, написанный Основателями...

--- Друзья мои, не будем ссориться! --- вернулась к ведению заседания Хранительница.
--- Если Совет не возражает, я проверю ещё гросс-другой заклинаний, не ограничиваясь одним в день...

--- Пустая трата времени и магии, --- вставил Нортог.

--- Отчего же? --- возразила Бореида, и бархатный смех мэтрэссы рассыпался по залу,
--- это убедит молодую девушку в том, что лотерея --- не лучший путь к вершинам науки!

--- Предлагаю голосовать, --- уронил Джованни.

--- А может быть, сначала нам новый портал покажут? --- спросил мечник.

--- С удовольствием покажу, друг мой, --- сказала Хранительница и направилась к двери.

Полминуты прошли в беззвучии. Лишь Катрин иногда проводила пальцем по крышке ноубука,
силясь найти закономерность в узорах трещинок.

\emptypar

А потом над столом вспыхнул небесно-голубой прямоугольник.

Нортог зажмурился.
Рудгерр рывком опустил забрало и заслонил глаза рукавицей.
Огонь --- хоть и был магическим --- жадно потянулся к кислороду.

Хранительница через портал протянула руку Катрин.

--- Поздравляю, --- улыбнулась она, пожимая ладонь беглянки.
--- Что бы это ни было, поздравляю.

За минуту, пока Мари закрывала портал и возвращалась с Центральной башни,
руку программистке пожали оба медикуса, ультрашархи, мечник, драконарий и наследник Жан.
Большей частью молча --- то, что было понятно, уже было сказано,
а пролить свет на неясное мог лишь эксперимент.

С перевесом в три голоса было решено провести опыт с гроссом заклинаний до конца квадры.

\emptypar
