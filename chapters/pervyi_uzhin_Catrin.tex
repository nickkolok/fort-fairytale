Кусочки жареной картошки одна за другой насаживаются на непривычно длинную вилку. Оп! Оп! Оп!
Хрустит корочка, сосредоточенно двигаются челюсти. <<Война - войной, а обед по расписанию>>.
А в голове трещит скоростной обсчёт, взгляд скользит между тарелок, впивается в узоры ткани, силясь понять: как?
Откуда берётся жареный цыплёнок, такой горячий, как будто только что приготовленный?

--- И всё-таки --- как это работает?

Во взгляде Катрин ясно читается ненасытное любопытство, которое не променяет желания поковыряться в сути вещей
ни на какую картошку, хоть жареную, хоть печёную, хоть с чёрной икрой. Хранительница беззаботно улыбается и пожимает плечами:

--- А твоя самобеглая карета как работает? И как она, кстати, называется?

--- Автомобиль, конкретно эта модель... Ну, фасон, сборка, тип, проект...
Короче, конкретно вот такое в обиходе называют <<копейка>>. Сейчас копейка - это монета мелкая... Медяк, да.
А работает довольно сложно --- двигатель там, поршни, цилиндры, потом кардан...

--- Вот скатерть-самобранка тоже сложно работает. Ткань с особым узором, заговор. Рецепты блюд.
Я, честно говоря, не теоретик. Вряд ли кто-то помнит, могу мэтров спросить на ближайшем Совете.
А вообще тут вещи веками служат; знаешь ли, есть такое заклинание --- нерушимость... на всякий случай куда попало накладывают;
оттого принцип действия того или иного артефакта зачастую бывает утрачен. Да, на стенах тоже нерушимость. Для них и делалась.
Прямое попадание катапульты, между прочим, держат! --- Хранительница на миг распрямилась,
глаза её азартно сверкнули --- и тут же погасли.

В глубине души она, конечно, хотела бы побывать на месте своих предков, которые много веков назад основали это убежище ---
квадрат земли, выгрызенный порталами и переброшенный в Южную Америку.
Островок свободы и надежды, который Сердце могло пристыковать к любой точке планеты.
Плоскую картинку, которую одинаково боялись и арабы, и инквизиторы.
Она вспоминала, как девчонкой, забираясь в библиотеку, бережно перелистывала рукописные ещё книги;
читать готические буквы было утомительно, и поначалу девочка занималась тем, чем занимаются обычные дети ---
просматривала картинки, на которых над гранитными стенами носились драконы, через ров летели пушечные ядра,
а из бойниц выглядывали арбалетные стрелы. Она представляла, как выхватив призванную шпагу,
прыгает с внезапно опустившегося моста к столбу над кучей дров и, прикрываясь щитами от стрел,
разрезает путы на очередном алхимике, приговорённом к сожжению, но вовремя воззвавшем о помощи.
Как мост, скрипя цепями, задирается торцом к небу, как Форт, соблюдая неписанный кодекс,
дожидается первого попадания лучников инквизиции по стенам --- и даёт ответный удар.
Как отрываются молнии от кристаллов Центральной башни, как летят арбалетные болты, как тает портал...
Обычно на этом всё и заканчивалось. Через пару веков после создания в Форте подобралось очень, очень научное ---
если в Средневековье вообще можно говорить о науке --- общество. При такой концентрации умов и немалых, в общем-то,
ресурсах магия рванула вперёд, ибо это было вопросом выживания.
Потом было прибытие других рас --- гномов, эльфов...
%TODO: разобраться с геологией. Ибо джунгли Амазонии с пещерами могут не вязаться.
Гномы, не имея возможности спариваться с людьми и особых способностей к магии,
занялись любимым делом --- горным ремеслом в пещерах. Так вокруг Форта появился ров, а стены прибавили вдвое по толщине,
втрое --- по высоте и впятеро --- по количеству пушек. Эльфы же, вполне способные к размножению между расами,
пополнили и без того мощную реку магических исследований. И уже через пару десятков лет после Исхода ---
как торжественно окрестили сумбурное мероприятие по экстренному вытаскиванию сотни ушастых страдальцев из горного ущелья,
в котором их зажали войска Ричарда Крепкой Печени\upcite{veresk_med} --- были устранены основные проблемы:
вырождение крови и болезни. Продолжительность жизни подскочила к ста годам...

--- Да ты кушай, кушай, --- Хранительница очнулась и вспомнила о роли хозяйки. --- Так намоталась, по горам, по долам...

--- Управление автомобилем требует не столько физической силы, сколько концентрации внимания.
Мне не есть, мне спать надо! --- рассмеялась Катрин.

--- Я уже распорядилась подготовить опочивальню, можешь отправляться...

--- А я не засну. Перевозбуждена. И любопытство всё-таки гложет...

--- И всё же пойдём. Теперь спешить некуда... Впереди вечность, позади вечность. Ты у нас вторая за сотню лет.
Ещё успеем всё обсудить. Пока подумай, что узнавать, куда копать... Ведь скорее всего, доступного объяснения нет.
Рецептов --- много, а теории... Есть, конечно, но в тех разделах, где приходилось создавать принципиально новые заклинания.
Гинекология, например; мы полностью победили смертность при родах... Это был один из важнейших вопросов после исхода.
У эльфиек крайне узкий таз, чем, собственно, немногочисленность этой расы и обусловлена.
Девочек у них исторически рождалось сильно больше, чем мальчиков, часто вообще двойнями...
А впрочем, ты ещё сама обо всём узнаешь... Про что мы говорили?

--- Про скатерть-самобранку, --- напомнила Катрин, и по голосу было понятно, что разговор с Хранительницей её немного убаюкал.

--- А скатерть-самобранка --- вещь простая! Они вообще делением размножаются. Растут потихоньку, если на них часто жир проливать,
а потом разрезать можно. Нет, --- рассмеялась она, --- специально капать на них не нужно.
Мы их при необходимости откармливаем. Сами обеспечены, гномов обеспечили, да ещё на складе лежат.

--- А она как-то готовит каждый раз блюдо где-то, да? --- Катрин отворила тяжёлую дверь из дерева, породу которого,
конечно, определить не смогла, но для приличия сочла дубом, и подошла к кровати.

--- Не знаю, дитя моё, не знаю...

--- Просто если она делает всё методом копи-пасты, то это открывает невероятные перспективы! ---
Катрин, залезшая было под одеяло, снова резким движением села. --- А если она готовит, то,
во-первых, интересно знать, где, во-вторых, непонятно, как она успевает ровно к моменту съедания предыдущего блюда,
да и откуда она берёт продукты?

--- Дитя моё, ты же сама говорила --- спать... Разве нет?

--- Просто я не успокоюсь, пока не поставлю парочку экспериментов...

--- Тщщ... Ложись на подушку... Сейчас будешь спать... Потом тебя обучат самоконтролю, а пока надо отдохнуть!
Для экспериментов нужна свежая головушка, --- Хранительница положила правую руку на лоб девушки,
а левой, запустив под одеяло, привычно нажала на пупок, ловя энерготоки.

--- ...Завтра ты проснёшься полной сил и свежей. Спокойной ночи...

Катрин провалилась в сон практически мгновенно. Ей снилось, что она спускается по лестнице ---
и на одной из площадок видит луг. Огромный луг с васильками. И озеро посередине него. Голубое, как васильки.
И она бежит среди васильков, прыгает в озеро, плещется в тёплой водичке, вылезает на берег, падая лицом в васильки,
и опять бежит купаться.

--- Надеюсь, я угадала, и плавать ты умеешь и любишь --- с тихой улыбкой проговорила Хранительница,
убирая руку с живота девушки. --- Спи сладко, тебя здесь никто не обидит...

Она встала и бесцельно побрела по коридорам жилой части крепости. А ведь насколько их поколение обмельчало!
Правду говорят --- на детях гениев природа отдыхает. Они сидят, читают книги, музицируют, медитируют,
несут рутинное дежурство... Кто-то из магов проводит всякие оптимизации пространственных скачков
и расчёты генераторных мощностей --- причём мало кто может объяснить, что это такое, но все знают, что это хорошо,
потому что помогает экономить и накапливать магию. А значит --- увеличивать квоты коацепции...
Первые три-четыре поколения основателей, по сути, основные проблемы решили. Еда изначально была в достатке;
медицину подтянули; когда плодовитость эльфиек, которые, обрадовавшись, начали предаваться любовным утехам лет с пятнадцати,
возросла уж слишком заметно, изобрели аккуратный способ предотвращать зачатие --- конечно, тоже магический.
Философский камень, как выяснилось, абсолютного бессмертия дать не мог,
а вот кардинально поднять продолжительность жизни до немыслимых в те времена ста-ста десяти лет --- пожалуйста.
Зато металл можно было получать, к ошеломлению гномов, из любого любой.
Итого Форт был обеспечен всем необходимым, ресурсов, за которые можно было бы соперничать, не осталось,
сигнал о помощи был позабыт во внешнем мире, и в некогда грозной крепости воцарилась сонная, дрёмная жизнь.

При которой никому и в голову не приходило задумываться: как работает скатерть-самобранка?

Хранительница, покружив по переходам, свернула в любимую библиотеку.
Пока беглянка Катрин отсыпается, неплохо бы поискать, что это за заклинание такое --- <<копий паствы>>...
