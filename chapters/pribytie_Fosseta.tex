
- И тем не менее, камень был брошен.
Эльфийка, истинный возраст который не смог бы определить, пожалуй, никто из присутствующих, поджала губы. Молодой хранитель, только недавно сменивший своего покойного отца, напряжённо молчал, оперевшись на стол Зала Совета и приглаживая усы согнутым указательным пальцем правой руки. Сначала левый, потом правый. Потом снова левый. И правый. Убедившись, что все высказались, он встал и обвёл глазами спешно собранный Совет.
- Как нам сообщил смотритель библиотеки, - мужчина кивнул на седого шархи, скромно притулившегося у стола - так, как будто бы, не будь стол круглым, он выбрал бы место у угла, - случай беспрецедентный. Предложение мэтрэссы Адельмины считать камень, брошенный в ворота Форта, нападением мне решительно не нравится. Предложение хирдового Рудгерра сделать вид, что нас тут нет и Форт - не более, чем заброшенная крепость, ещё хуже. Они ведь могут вернуться - и привести большую экспедицию.
- Вернуться? - скрипучий смех коренастого гнома наждаком прошёлся по стенам зала, - да их всего трое. Старик и двое раненых. В сельве.
- Мы не знаем, откуда они пришли. Мы не знаем, на чём. Мы не знаем, могут ли они как-то сообщить о своём местонахождении. Мы не знаем, кто они и зачем пришли, в конце концов!
Хранитель остановился, чтобы перевести дыхание. Эльфийка перебирала бусы. Гном сжимал рукоять топора на поясе. Библиотекарь весь обратился в слух и зрение, желая как можно точнее ухватить детали исторического момента с целью последующего описания.
Наконец, в звенящей тишине раздался голос первой наследницы - дочери Хранителя, вызванной им на совет по старой традиции - да и вообще, чтобы привыкала.
- А почему бы просто не спросить?
- И как ты себе это представляешь? - Хранитель, который сам хотел было высказать эту идею, делано удивлённо уставился на девушку.
- Спустить мост не до конца да поговорить. Или у нас эмпаты перевелись?

- Вот и пришли.
Старый полковник пригладил измочаленную, грязную бороду и, подобрав с земли небольшой серый камень, с трудом бросил его в деревянную часть стены.
- Отец, а ты уверен, что нам ответят?
- Мне всё равно, Джек. В любом случае я добьюсь того, ради чего жил последние двадцать лет; первую часть я уже знаю: крепость есть, и она не разрушена. Нет ни следов битвы, ни человеческих костей вокруг неё - следовательно, либо она обитаема, либо последние её жители скончались внутри... Но мост не прогнил. И вода во рву похожа на проточную, что, впрочем, можно приписать какому-нибудь источнику...
Рэли закончил возиться с костром. Пламя радостно лизало сухие палки, пуская в небо струйки дыма - ещё один шанс путников быть замеченными. Крепость безмолвствовала.
- Отец, а если нет?
- Пойдём назад, сынок. Пройдём. Бросим в ближайшую реку бутылку с описанием.
Все трое понимали: кроме бутылки, которую могут и не найти, от них вряд ли что-то останется. По мере того, как они продвигались к намеченной цели, уверенность Персиваля в том, что направление выбрано верно, крепла по мере того, как нарастала опасность. Сначала - кровожадные индейцы, от которых путешественники с большим трудом откупились; потом - странные земляные укрепления, построенные, судя по безмятежному зелёному месиву обжившей их растительности, много лет назад; полоса болот, для этой местности вполне характерных, но каких-то подозрительно кольцевых. Малярия, хищники... Если бы не талисман, передававшийся из поколения в поколение и сейчас висевший у полковника на шее - не дошли бы.

- Кроме того, - почувствовав поддержку, продолжал Хранитель, - последний беглец был лет сорок назад...
- Тридцать восемь, - вставил библиотекарь.
- ... и оказался каким-то шаманом с Крайнего Севера, вообще из России. Как он умудрился нас вызвать с помощью бубна и семи черепов оленя - ума не приложу.
- Случайно вызвал похожее возмущение астрала, - пожала плечами Адельмина.
- Предпоследний - больше полувека... Франция. Бежал от Революции. Я хочу лишь сказать, что, по всей видимости, ритуал вызова во внешнем мире утрачен, и способов узнать, что там происходит, у нас почти не осталось. При всём уважении к сноходцам, - он посмотрел на шархи, - из их сводок мы понимаем мало. Самобеглая повозка, работающая на угле - она действительно существует или это кому-то приснилась печь из сказки про этого... Емельяна Пугачёва?
- Просто Емелю, - сказал шархи, - Пугачёв - это не сказка...
- Зеркала - огромный расход энергии при полном отсутствии понимания происходящего. Речевые чары сквозь стекло не работают; в созерцании спящих или трапезничающих людей особого смысла нет... Да и вырождение крови может дать о себе знать. Старик уже мало на что годен, а вот ребята...
Адельмина вздохнула. Трое дочерей росли, а юношей всегда не хватало. Эльфийская наследственность...
- В связи с этим предлагаю обсудить условия, на которых Форт будет готов их принять. В частности...

...Прошёл томительный час. Цепи заскрипели, мост пришёл в движение. Медленно, неохотно пробуждаясь от многолетнего покоя, дощатая пластина опустилась - и замерла, не дойдя семи футов до внешнего берега рва, как раз так, что, даже подпрыгнув (а на вывернутой ноге не очень-то попрыгаешь), нельзя было за неё уцепиться.
Обитаема.
Перси попытался поправить лохмотья, в которые превратилась его одежда, и, с трудом встав, подошёл к точке рандеву.

По мосту застучали шаги. Молодой, лет тридцати на вид, мужчина в длинной чёрной мантии подошёл к краю первым, и на солнце тускло блеснул золотой обруч, державший его волосы; следом за ним, чуть позади и справа, показалась девушка лет тринадцати; её причудливая причёска, слегка покачиваясь, держалась на таком же обруче, но серебряном; она не куталась в мантию, наоборот, лёгкий кожаный... корсет, что ли?.. оставлял открытой её талию, закрывая, однако, плечи.

Следом, слева и вплотную к мужчине, который, судя по всему, был главным в делегации, шла босая девушка в одеянии ещё более открытом; из-под распущенных волос игриво торчали кончики длинных ушей, взгляд нечеловечески больших карих глаз был направлен одновременно никуда и всюду.
Слева от неё встал карлик, закованный в сверкающие доспехи, с топором на поясе; на правый край вышла, опираясь на посох, женщина с такими же острыми ушами, как и у босоногой, но в серой, немного потрёпанной мантии.
Фоссетт не стал подходить близко, не желая задирать голову; члены маленькой экспедиции встали напротив моста - посередине полковник, справа - сын, слева - Рэли...
- Приветствуем вас, путники! - прогремел бас Хранителя, - кто вы и зачем пришли?

- Полковник Персиваль Фоссетт, подданный Его Величества Георга Пятого, короля Британии, императора Индии, короля доминионов и колоний. Мой сын Джек Фоссетт и его товарищ Рэли Раймел, также подданные Его Величества Георга Пятого. Мы пришли, чтобы узнать правду.

Босоногая, дождавшись окончания его речи, едва заметно тронула главного за руку. "Не врут, - понял Хранитель, - а толку с того?"
- Пришли ли вы за знаниями ли за вещами? - заговорила женщина в мантии.
- За знаниями. И мы их уже получили. Город Z перед нами.
Ещё одно прикосновение ладони девушки-эмпата. Отлично.
- Что вы намерены делать дальше?
Молчание. Пять секунд, десять.
- Мы не знаем... Попытаемся вернуться.
- Не лукавьте! - прозвенел голос босоногой.
- Мы не надеемся вернуться.
- На что вы в таком случае рассчитывали?
- Что сумеем вернуться.
Снова пат. Хранитель думал, стараясь не показывать своего волнения.
- Будут ли вас искать, если вы не вернётесь?
- Нет, - старик ответил мгновенно, - я написал распоряжение.
- И вы уверены, что оно будет исполнено? - голос карлика из-под опущенного забрала звучит угрожающе и насмешливо.
- Нет. Но если и будут - не найдут. Я сообщил ложный маршрут. Здесь - не найдут.
- Готовы ли вы ступить на землю Форта, став полноправными его жителями, отрезав себе путь к возвращению?
- Нам надо посовещаться.
- Вы ведь уже всё решили, полковник, - опять босоногая, мысли она читает, что ли? - и ваши спутники. Но сообщить об открытии вы не сможете.
- К чёрту сообщение! - вспылил Фоссетт, - Обрыдло...
- Повторяю, - опять мужчина по центру, - готовы ли...
- Готовы!
Вопросительные взгляды оценивающе пронзают молодых людей.
- Готов.
- И я...
Мост медленно и аккуратно опускается до конца. Цепь размыкается.
- Добро пожаловать!


...В камине горит огонь. И ничего не болит. Какое же это счастье, когда ничего не болит!

Джеки что-то лопочет о Нью-Йоркской бирже, о моде, о каких-то законах. Рэли поглощает жаркое. И старика ничто не отвлекает от осознания собственной победы.
Тарелка пустеет за тарелкой, на столе откуда-то берутся новые и, наконец, их собеседник, представившийся Хранителем, задаёт тот самый вопрос:
- Как Вам удалось нас найти, Персиваль?
Полковник усмехнулся, погладил бороду...
- Первая зацепка - амазонки. Белокожие женщины, в бой с которыми вступил де Орельяна. Сам факт их упоминания ничего не говорил о городе, но зато наводил на мысли о крайне необычном государственном образовании. Образовании, имеющем европейские корни.
Дальше - больше. Откуда здесь столько золота?
Наконец, легенды индейцев. Есть, мол, затерянный город, и найти его нельзя, но приходят оттуда белые люди и меняют золото на другие металлы - в бОльших, естественно, количествах.
В пересказе дикарей это и не люди вовсе, а божества, умеющие исцелять и перемещаться в пространстве.
Ничего похожего на действительно могущественное государство так и не нашли, потому что искали там, где эти полубоги появлялись.
Болваны... Зачем вам открывать своё истинное местоположение? Я просто нанёс на карту все точки, где индейцы рассказывали о встречах с белыми людьми и чудесах. Или где белых людей уважали и почему-то поклонялись им. И отправился ровно в центр "белого пятна", в котором отметок не было.
Уже по пути корректировал маршрут, стараясь выбирать наиболее сложный путь...
Хранитель покачал головой.
- Есть ли шанс, что кто-то его повторит?
- Я сжёг рукописи.



