Вот и настало время твоей последней сказки. Самой главной. И малейшая капля лжи в ней означает смерть...


 - Принят сигнал о помощи! Все на стены!
Хранитель кладёт руки на кристалл, собирая мозаику образов, пронзающую пространство. Сигнал какой-то странный, как будто адресованный не им... Впрочем, если уж кто-то решился вызывать Форт, логично предполагать, что этому кому-то очень несладко.
 - Вы тоже заметили? - подозрительно прищурилась Адельмина.
Мужчина коротко кивнул, разрывая ткань пространства.
 - Джек Фосетт и Рэли Раймел, поднимитесь на Центральную башню! - разнесло заклинание его голос.


Есть! Вот он, корабль, почти на горизонте. И, набирая ход, стремится к линии, отделяющей небо от воды.


 - Это самолёт, - уверенно ответил юноша. - И он явно горит.
 - Сигнал исходил от человека в нём, - мэтресса сосредоточено сжала губы.
 - И, вероятно, был адресован кораблю - махнул рукой Хранитель, указывая куда-то на простор Атлантики, открывавшийся за чертой портала.
 - Либо пилот собрался поджечь корабль, упав на него на горящей машине, - начал Джек, - либо... нет, за двадцать лет... Никогда не слышал, чтобы самолёт сел на палубу парусника. Боюсь, что это невозможно.
Хранитель медленно моргнул. Самолёт и корабль уже почти скрылись из виду. Прыжок портала...
 - Предлагаю тушить, - заговорил Рэли, и по голосу юноши было понятно, что он считает пилота благородным безумцем. - Брандер из обугленной "этажерки" никакой. А если он хочет сесть - то хоть корабль не сожжёт.
Джек хотел было поднять посох, но Хранитель его опередил. "Как многого я всё-таки не умею..." - подумал парень, наблюдая, как Хранитель сделал шаг в сторону, положив руку на один из угловых кристаллов площадки, а потом... Фигура мужчины буквально на миг дрогнула, полупрозрачный силуэт вскинутого посоха мелькнул перед глазами Джека и Рэли... А маг продолжал стоять, как будто и не двигался.
Короткие белые молнии стайками отрывались от кристалла, легко догоняя самолёт.


Решительно, это везение. Ты уже отчаялся сбить огонь, как вдруг... Не расслабляться! Бриг уже близко. Он умён, очень умён: набрал скорость, подстраиваясь.


 - Всё, что мы могли, мы сделали, - рассеянно вздохнул Хранитель. - Остаётся только наблюдать, - и добавил, - Дракон его не догонит.
Мысли его были вовсе не о странном человеке, отчаянно борющимся за жизнь с таким чуждым небом, нет... Доктрина защиты Форта, основанная на том, что в воздухе нет врагов, зато есть драконы, летела в тартарары. Конечно, экспедиция Фосетта рассказывала об этих летающих конструкциях, но аэростату хватило бы одного удара драконьего хвоста... Да и, если уж совсем честно, маги тешили себя надеждой, что люди поиграются-поиграются, да и бросят мысль о полёте.


Шасси наготове. Корма. Слева проносится мачта... Удар в крыло. Остатки самолёта вот-вот вылетят за борт... Но нет. Такелаж будто бы нарочно сплетён так, чтобы этого не произошло. Странно, что он не порвался, но это уже детали.


Сигнал класса тау-один сотряс астрал, заставив несколько сотен адептов Форта вздрогнуть и вскинуть головы к небу.
Такой сигнал всегда означал смерть. Смерть конкретного человека.
 - Со мной всё в порядке, - разнесла магия голос Хранителя, - я жив и здоров. Это корабль принял нового Капитана.

