
/* Вообще по атмосфере этот кусок должен быть выше, но уже после того,
как Катрин вычислила малые порталы и притащила Громостонко */


Катрин отложила в сторону ножик и смахнула содержимое тарелки в вазу.

--- И всё равно не понимаю, что ты в этом находишь, --- вздохнула Линдариэль,
лёгким движением пальцев подвешивая огурец в воздухе.

--- Наверное, это потому, что я линуксоид, --- улыбнулась Катрин, --- надо ж мне хоть что-то собрать из исходников!

--- Правильно, --- поддакнул Громостонко, отправляя очередное творение скатерти-самобранки за щёку, --- в этом есть романтика!

Наставительно поднятый палец беглеца-доцента несколько дисгармонировал с набитым ртом.
Эльфийка ничего на это не ответила --- романтика так романтика ---
тонкие пальцы левой руки юной волшебницы проделали замысловатый пасс, и огурец, накрошенный пирамидками, упал в тарелку.

--- Иван Алексеевич, Вы нашли картошку и горошек? --- спросила Катрин.

--- Картошки в ней нет, --- со знанием дела, как будто он не находился в магическом убежище в лесах Амазонки,
а участвовал в застолье на родной кафедре, --- и быть не может. Потому как делали её в Европе в невесть каком году.
Есть какая-то репа. Горох есть. Варёный, не консервированный.

--- Репа, как Вы выражаетесь, сойдёт, --- вынесла вердикт Катрин после краткой дегустации.
--- Горох тоже. Теперь самое сложное --- майонез...

--- Успеем, ещё пять часов... Мы ведь по Гринвичу? --- уточнил Громостонко.

--- Ясное дело, как истинные космополиты и борцы с системой, --- улыбнулась Катрин.

--- Борцы... --- тихо проговорила Линдариэль, --- неделю назад отметили солнцестояние, какой смысл-то?

--- Вот не надо лишать нас праздника, --- подал голос Громостонко.

Он в самом начале приготовлений заявил, что <<топология хорошо развивает воображение>>
и поэтому он будет пытаться вытащить нужные ингредиенты из скатерти-самобранки, и вообще, человеку,
который дважды в год на последней пересдаче вытягивает из двоечников, которых нельзя отчислять,\upcite{FGOS-2025} то,
что эти самые двоечники никогда не знали, нет ничего проще, чем...

--- Лишь бы салат не резать, --- прервала его тогда Катрин.

Но нужно было признать: с задачей он справился.

--- И вообще, если тут кому-то наша инициатива не нравится --- может не отмечать, ---
сурово взглянул он на рыжеволосую эльфийку, --- лично я тут никого не держу.

--- Да нет, --- пожала бледными плечами девушка, --- мне просто интересно.

--- Для полного антуража нам не хватает курантов и обращения президента, --- ухмыльнулась Катрин.

--- На что это Вы намекаете? --- удивился Громостонко, --- лично мне не слаб\'o за Мари сбегать...

\emptypar

\emptypar
