

--- В конце концов, люди ждут чуда, друзья мои, --- проговорила Хранительница, --- так какая разница, что им явить?

--- Значит, как обычно: я икс, мэтр игрек, Линдариэль зэд? --- спросил Громостонко.

--- Именно так. Спешу открыть портал. Катрин, отвлекись от сводок --- путеводствуй!

\emptypar

...Пламя подступало к Ляпуновску со всех сторон. На севере и востоке полыхали цистерны нефтеперерабатывающих заводов,
на западе --- сосновый лес, насаженный некогда <<для оздоровления города с химической промышленностью>>,
а на юге смрадно чадил <<Ляпуновсксинтезкаучук>>.
Местное телевидение каждый час рапортовало об успехах пожарников,
да вот только не видать что-то этих успехов из окна шестнадцатиэтажки.
Мария Сергеевна горько усмехнулась, проклиная аномально жаркое лето.
Написать, что ли, в <<Яндекс.Соловей>>, заменивший запрещённый твиттер? Приедут? Да пусть приезжают.
Всё равно задыхаться. Или не стоит...

...Из-за дыма никто не видел, как далеко в небе показалась серая точка.
А потом внезапно прямо над зданием вокзала повисла классическая летающая тарелка.
Она отнюдь не впечатляла размерами --- серебристый диск метров пять, максимум десять в диаметре,
с кругляшками иллюминаторов --- но из окон близлежаших высоток, несмотря на дым, была различима прекрасно.
НЛО медленно поднялось вверх, поравнявшись с крышами самых высоких домов.
Тонкий зеленый луч прочертил небо, соединяя иллюминатор с горящим бором.

Минута прошла в томительном ожидании. Спасают или добивают? Свои или чужие?
ли это вовсе всего лишь плод сознания, воспалённого парами химикатов? А потом лес стал гаснуть.
Медленно и плавно, словно кто-то крутил невидимую ручку газовой конфорки. Ещё один луч ударил в каучуковый завод...

...Всё было кончено через четверть часа. Ветер разгонял осиротевший дым, серебристый диск, сделав круг почёта,
умчался за горизонт.


\emptypar


--- Надо сказать, по энергии мы даже в плюсе, --- бросая искоса серый взгляд на старшего мага, промолвила Катрин.
--- Нефть ох как хорошо горит...

--- Да и тот элемент, облако которого мы просто поглотили в резервы --- он у вас называется <<бром>>, так?
--- тоже далеко не лишний. Теперь, дитя моё, мы им на долгие годы вперёд обеспечны с лихвой...


\emptypar


\emptypar







--- <<Никакого НЛО над Ляпуновском не было>>. ИТАР КЛАСС, 16:50.

--- <<Особую доблесть при тушении сложнейшего возгорания проявили бойцы противопожарных войск --- губернатор>>. Крабио, 16:43.

--- <<Разлившиеся химикаты могли сыграть роль галлюциногенов>>. Первый канал, 17:37.

--- <<Все разговоры об инопланетянах --- провокация и ложь!>>. ИТАР КЛАСС, 17:20.

--- Без проблем! <<Жителям Ляпуновска, пострадавшим от утечки галлюциногенов,
будет оказана наркологическая и психиатрическая помощь>>. Третий канал, 18:45. Тебе на Щ.

--- Вот как, значит? --- Линдариэль откинулась на спинку каучукового кресла, взглядом вращая колёсико мыши,
--- <<Щель в обшивке трубопровода могла оказаться причиной аварии на Ляпуновском СК>>. Крабио, 17:32. Тебе на К.

--- <<Конфедеральное министрество безопасности: выдумки жителей не окажут влияния на результат работы экспертной группы>>.
Зеркало Рунета, 17:11.

Пауза. Эльфийка сосредоточенно морщится, тыкает взглядом то в клавиши, то в кнопки мыши.

--- Линда, нет слов на букву <<Ы>> в русском языке. Мы же говорили, --- улыбнулась Катрин,
--- а уж научить тебя слепой печати я вообще отчаялась. Да и сама, честно говоря, не сторонница...

--- Значит, на П. <<По факту разлива нефтепродуктов под Ляпуновском возбуждено уголовное дело>>. Русновости, 16:32. Вот!

--- Та-ак, это уже было... <<Опасения по поводу инопланетного вторжения абсолютно беспочвенны>>.
Второй канал, 18:08. Тебе Ы, для закрепления знаний.

--- <<Над Ляпуновском не находилось посторонних летательных аппаратов>>, --- улыбнулась эльфийка, --- ИТАР КЛАСС, 18:52.

--- <<Временный приют в подвалах местных школ получат ляпуновцы, потерявшие жильё>>. Первый канал, 18:44.

Снова телекинетическая пляска на клавиатуре. Вздох. Хихиканье.

--- <<Ёжики размером с человека, летающие тарелки и фиолетовые гоблины, или Бойтесь бромбензола>>. Новости культуры, 17:32.

--- <<Абсолютно бессмысленными назвал министр безопасности рассуждения об НЛО в ходе экстренной пресс-конференции>>.
Конфедерация-24, 19:20.

--- <<Инопланетян не существует...>>

\emptypar

--- Дети мои, как ваши успехи? --- спросила Хранительница, входя в спецлабораторию, --- Что удалось узнать?

--- Знаете, --- покачала головой Катрин, --- если б мне кто-нибудь лет шесть назад заявил,
что я буду обучать эльфийку русскому языку и информатике одновременно, сидя в бункере посреди амазонской сельвы,
да ещё и на примере новостей о собственноручно сфальсифицированном НЛО... Отправила бы на это, как его...

--- Оказание наркологической и психологической помощи, --- подсказала ей компаньонка.

--- Вот-вот. А по делу... Линдариэль делает успехи. Пропаганда Конфедерации делает вид, что ничего не было.
Но шуму мы наделали много.

--- Значит, продолжаем вылеты, --- одобрительно улыбнулась Хранительница,
---  Едва ли чаще, чем раз в два дня, чтобы не отвлекать мэтра и доцента от работы и не набить оскомину внешнему миру.

--- С максимальным пролётом над населёнными пунктами, --- уточнила Катрин,
--- и только на крупные пожары. Чтобы всё-таки оставаться по энергии в плюсе.

--- Дитя моё, конечно, в настоящее время тушение огня во внешнем мире неплохо пополняет наши запасы, но во-первых,
для получения, как ты изволишь выражаться, <<плюса>> достаточно открывать небольшой портал и тушить из него,
не гоняя этот бублик-самолёт или как вы его там называете, во-вторых, --- женщина села на третье кресло, расправив мантию,
--- в качестве постоянного источника использовать случайные происшествия во внешнем мире не подобает.
А потому наша главная цель --- попадать в поле зрения людей,
проявляя себя при этом с как можно более героической стороны и не обращая особого внимания на баланс магии.
Продолжайте отслеживать хроники.

\emptypar

\emptypar
