Солнечный луч заглянул в окно и побежал по стене, отражаясь от редких пылинок, отмечавших его след.
Катрин приоткрыла глаза, готовясь выдать стандартное для студента <<мне ко второй>>, и внезапно поняла, что выспалась:
что ей не хочется зарыться под одеяло, засунуть руки под подушку, прижимая её к лицу, или долго метаться по кровати,
выбирая то самое положение, которое позволит отоспаться за пятнадцать минут лучше, чем за предыдущие шесть часов сна.
Понимание это, для студентки мехмата совершенно противоестественное, насторожило её пробуждающееся сознание;
мозг, включаясь непривычно охотно, заявил, что окно в её комнате находится совсем с другой стороны.
Девушка осторожно приподнялась на кровати, заметив,
что спала в одежде и что одежда эта явно нуждается в стирке после трудного дня.
Когда её взгляд пробежал по гранитным стенам и нащупал часы, на которых вместо двенадцати делений было восемь,
и те без чисел, память наконец потянулась, зевнула и приступила к выполнению своих обязанностей.

После того, как ответ на вопросы <<Где я?>> и <<Кто я?>> были найдены
(а в ответе на вопрос <<Кто виноват?>>\upcite{kto_vinovat} она, как и всякий линуксоид, не сомневалась ---
проклятые копирасты, цензоры, виндузятники и яблочники), во весь рост встал вопрос <<Что делать?>>.\upcite{chto_delat}

А действительно, что? Допустим, одеваться не надо --- она не обнажена.
Из сменного, правда, только купальник в машине, но, судя по тёплой встрече, какую-нибудь рясу ей здесь выдадут.
А если разрешат, можно и перешить хоть в мини-юбку. Линуксоид пересоберёт всё, что угодно.

Гораздо хуже обстояли дела с тем, чтобы умыться, почистить зубы и совершить ещё одно утреннее дело,
которое в сказках обычно не упоминают;
Катрин уже собралась пошарить рукой под кроватью, отыскивая необходимый, так сказать, артефакт,
но тут дверь бесшумно отворилась, и в комнату вошла Хранительница.

--- Доброе утро, дитя моё! Надеюсь, ты отдохнула?

Катрин стушевалась от подобного внимания; всё-таки в представлении человека, привыкшего к субординации ---
 не столько даже академической, сколько бюрократической --- внимание коменданта крепости к беженцу ограничивалось приветствием,
 при определённом везении --- выдачей документов, в лучшем случае --- обедом, и то не тет-а-тет.

--- Спасибо... Всё хорошо, не ст\'oит... Что же, Вы всё время ждали, пока я отосплюсь?

Хранительница с тихим пониманием улыбнулась:

--- Отчего же? Не всё в этом мире столь прямолинейно, а уж в Форте и подавное многое встаёт в ног на голову.
Я \textbf{знала}, когда ты проснёшься. И я знаю, что тебе снилось.
А общаться с новоприбывшими --- это моё прямое обязательство, по которому я, признаться честно, соскучилась...
Да что там <<соскучилась>> --- только в книжках читала и с нетерпением ждала случая исполнить!

--- Мне б умыться как-нибудь, --- осторожно вставила девушка.

--- Дверь в соседней стене, дитя моё. Верно ли я предполагаю,
что к сему году цивилизация внешнего мира достигла достаточного уровня,
чтобы каждый её представитель сумел совладать с механическим управлением водными потоками?
Или же мне придётся, подобно моим дальним предкам, смутить тебя незамысловатым наставлением?

--- Думаю, справлюсь сама, --- заулыбалась, садясь на кровати, гостья.

\emptypar

Через пару минут Катрин уже задумалась о том, что краткий инструктаж был бы весьма кстати. А то ведь и не погуглишь!
Хоть мануал какой-нибудь почитать, что ли. А тут не то что мана --- тут даже освежителя воздуха нет!
С местным аналогом крана она разобралась довольно быстро: как выглядит алхимический значок воды, когда-то где-то видела;
верхний кристалл-ползунок на стене, двигавшийся между никак не обозначенным концом и этим самым значком, повторённым трижды,
как верно догадалась девушка, регулировал мощность потока, бившего прямо из стены во внушительных размеров каучуковую ванну,
не имевшую слива и для чего-то спрятанную под тонкой металлической крышкой, сделанной, судя по весу, либо из алюминия,
либо из некоего сплава с его участием.
Нижний ползунок перемещался между снежинкой и костром и, очевидно, регулировал температуру.
Умывшись и почистив зубы по-походному --- пальцем, Катрин стала искать полотенце --- и не нашла.
Зато обнаружила аналогичную группу из двух кристаллов под руной ветра --- магический аналог фена.
Девушка закрыла за собой ванну крышкой (раз так было --- значит, так и надо оставить) и попыталась найти унитаз;
ещё раз окинув комнату взглядом, она пришла к выводу, что его роль исполняет каучуковое кресло;
вниманительно осмотрев подлокотники, девушка обнаружила на левом --- регулятор от солнышка до кружка,
а на правом --- пару уже привычных ветра и воды.
Солнышко на поверку оказалось точкой, а сдвиг кристалла открывал собственно горшок, не имевший, как и ванна, слива.
Терпения на дальнейшие рассуждения у беглянки не оставалось, и ей пришлось довериться интуиции.
Сжав отверстие в сиденьи обратно в точку, Катрин отправилась к ванной, чтобы помыть руки;
подняв крышку, девушка обнаружила, что ванна сияет чистотой и сухостью.
Вернушись к креслу-унитазу, беглянка подвигала диаметрический кристалл --- и обнаружила внутренности агрегата кристально чистыми.
Вымыв и высушив руки, она решила, что уплатила гигиене достаточную дань.

...В камине, спокойно и уверенно хрустя дровами, горел огонь. Завтрак мало чем отличался от обеда или ужина ---
скатерть-самобранка имела в распоряжении, конечно, достаточно большую <<кулинарную книгу>>,
но старалась ориентироваться на образы, возникающие у едока, поэтому Катрин, попытавшись получить йогурт,
последовательно отставила в сторону какую-то доисторическую ряженку, сыр с плесенью и остановилась на десерте,
который (для солидности) про себя назвала <<Тирамису>>.
Хранительница --- то ли из вежливости, то ли по каким-то психологическим соображениям,
то ли с простой и обыденной целью подкрепиться --- лёгким прикосновением к ткани извлекла то же самое и повторила вопрос.

Катрин кивнула веками: <<Думаю>>.
Ещё минута прошла за поглощением десерта, и наконец девушка заговорила:

--- За распространение книг по интернету. Учебников.

--- Ты хочешь сказать, что общество, решившее проблему безмускульного передвижения, до сих пор борется с книгами?!

--- С недавних пор --- да. Всё началось лет десять назад... С чего бы начать... А можно нескромный вопрос?

--- Ты всегда вольна спросить, но...

--- ... не всегда получу ответ. Помню. Так вот, Вы говорили о том, что за последние сто лет нас прибыло двое --- я и Зайка.
Вернее, наоборот --- Зайка и я. А кто был перед нами? Мне просто нужно ориентироваться, на что можно ссылаться...

Хранительница склонила голову на бок и коротко ответила:

--- Фоссет.\upcite{vals_gemoglobin}

--- Отлично. Так вот, про телеграф Вы знаете? Его развили. По проводам, а после и вовсе по воздуху.
Да, без магии, на одних радиоволнах.
И не только книги -- ещё фотографии, записи музыки, синематографические произведения, которые называют фильмами.
Поначалу люди радовались и старались связать как можно больше компьютеров...
Это такой особый аппарат, на котором можно делать многое, в том числе --- читать книги и смотреть фильмы со звуком,
теперь самые компактные из них умещаются в часах...
А обычно экран --- та часть компьютера, на которой меняется изображение --- размером примерно с раскрытую книгу.
Так вот, люди построили огромную сеть связей между компьютерами --- Великую Паутину.
Нет, каждый не обязательно соединялся с каждым; постепенно выделились специальные компьютеры-посредники.
Примерно до двухтысячного года от Рождества Христова...

--- У нас совпадает летоисчисление, --- кивнула Хранительница.

--- ... все радовались, что могут, не выходя из дома или зайдя в ближайшее помещение,
за плату предоставляющее компьютеры на время тем, у кого не было собственного, посмотреть фильм или прочесть книгу,
которых не то что в городе --- в стране нет.
Но потом постепенно издатели книг стали скупать у авторов патенты... Персиваль ведь наверняка рассказывал?.. патенты на текст.
И запрещать бесплатно распространять патентованные книги.
Потом то же произошло с фильмами. Сначала никого особо не наказывали, а потом, в 2012-м...
Тогда ещё Российская Федерация --- один из осколков Союза, самый крупный --- заявила,\upcite{fz89417-6}
что через эту сеть распространяются фотографии голых девочек, наркотики и инструкции для самоубийц,
поэтому его нужно взять под контроль. И некоторые куски информации запретить, ради блага детей.
Потом ввели запрет и на распространение патентованых произведений искусства.
Сначала просто разрывали связь с распространяющим компьютером,
лишали его права на запоминащийся адрес --- так называемое доменное имя --- а потом начали сажать.
К 2022 году дошло до пожизненных сроков --- <<за угнетение прогресса и преступление против культуры человечества>>.
Пройдохи, вовремя скупившие патенты --- которые стали называться лицензиями, озолотились, а тех,
кто распространял или скачивал книги нелегально, не имея патента, ещё с <<нулевых>>...
то есть с 2000-х годов... стали называть пиратами.

--- А владельцев патента --- капёрами?

--- Вы почти угадали, --- рассмеялась рассказчица, --- копирастами. Задолго до того, как копирасты подняли голову,
в 1980-х, появился странный, чудаковатый философ --- Ричард Столлман.
Он предсказывал многое из того, что случилось потом.\upcite{Stollman_earliest}
Его мало кто слушал, но нашёлся студент-компьютерщик Линус Торвальдс --- из скандинавов --- который придумал операционку...
Программу... В общем, способ настройки компьютера, патентом на который обладало всё человечество.

--- Дитя моё, прости, что я вынуждена перебить твоё повестование, но твоё стремление объяснить мне быт внешнего мира словами,
понятными для жителя начала двадцатого века, конечно, заслуживает благодарности,
однако я предпочла бы попросить тебя сменить способ изложения, если, конечно, ты не возражаешь...

--- Но... я плохо умею рисовать, --- засмущалась Катрин.

--- Устные способы общения не исчерпываются одной только речью, ---
улыбнулась Хранительница и дотронулась рукой до тонкого, почти неприметного золотистого обруча,
на висках выступавшего из-под уже чуть тронутых сединой волос, --- ты позволишь мне заглянуть непосредственно в твою память?

Катрин рассмеялась:

--- Конечно. Знаете, после женкомата меня уже не напугать заглядыванием куда бы то ни было.

\emptypar
