%TODO: Катю заменить на что-нибудь незанятое. Например, на Надю.

%TODO: Переписать. Безопасник не ищет, а только проверяет ранее найденное.


Привычно мягкие подлокотники. Красный бархат кресла в первом ряду. Каблучки выпускниц-девятиклассниц, стучащие по сцене...
Полковник безопасности без особого труда подавил желание расслабиться. В конце концов, посмотреть на девочек он может всегда - у него-то есть доступ ко всему, что сосёт данные из ипотечных квартир. А здесь он при исполнении...
 - Что скажете, Ваша Бдительность?
С обращениями для сотрудников их ведомства - явно перебор. Монархизм какой-то. Полковник жестом показал директору гимназии на кресло рядом с собой, тот осторожно сел.
 - Судя по репетиции, почти всё в порядке...

"За исключением множества меры нуль," - мстительно добавил в голове голос лектора, читавшего в училище функанализ.
Надо же, сколько лет прошло... Повышенное восприятие, чтоб его!

 - ... но куплет про глобус, стоящий на книжках, лучше всё-таки убрать. Знаете ли, пропаганда превосходства глобалистических ценностей перед наследием русской классической литературы... Кто автор?

 - Берёзкина, - вздрогнув, сказал директор, и поспешно добавил: - Но Вы не волнуйтесь, золотой медали она через два года не получит...
 - Ну зачем же так, - улыбнулся полковник. Он всё никак не мог привыкнуть к пузатым пятидесятилетним мужикам, лебезящим по поводу и без. И каждый раз получал от этого специфическое удовольствие, вспоминая директора своей школы. - Вызовите её ко мне.
 - Берёзкина! - срывающимся на визгливый крик голосом заорал директор гимназии. - Живо сюда!
Девушка, подобрав белое платье, спешно спустилась со сцены.
 - Звали, Фёдор Иванович?
 - Ещё как звал! Да ты знаешь, что ты натворила? Отвечай за свою писанину!
 - Не нужно делать за меня мою работу, тем более, делать её неправильно, - полковник поднялся с кресла, бросив свинцовый взгляд на директора. - Екатерина, подойдите ко мне.
Школьнице ничего не оставалось делать, как подчиниться. Безопасник осмотрел её с ног до головы (именно в таком порядке проведя быстрым взглядом) и внимательно посмотрел в глаза. Не то. Пустышка, очередная из тысяч.
Нет, если бы его целью была раскрываемость, пожалуйста, можно брать. Но он пришёл сюда не за этим...
 - Екатерина, сразу после прогона - со мной на воспитательную беседу. На медаль идёте?
 - Х-хотела бы...
 - Плохо хотела! - влез директор.
 - Небось ведь сами попросили бедную девушку написать сценарий?
Катя кивнула.
 - Вот видите... Прогоняем выпускной ещё раз, уже без этого куплета. Надо бы ваши пассажи повнимательнее изучить...
Девушка, чуть не плача, убежала за кулисы, а полковник снова сел в кресло и откинулся на спинку. Надо бы ещё какой-нибудь повод придумать, вдруг всех изучить не успеет...


Когда девятиклассники в последний раз пропели про глаза, которые надолго разлучаются, полковник тяжело вздохнул. Здесь тоже никого. День, можно считать, прошёл зря. Возвращаться в Москву из командировки с пустыми руками не хотелось... Но в этом райцентре каких-то пять школ. Два прогона часа по полтора на каждую плюс зазоры... Итого два убитых дня. Два дня из десяти-двенадцати, что есть на эти странные проверки каждый год... Устраивать совсем уж явные поиски не решались. Кто знает, чем может кончиться излишнее внимание?
Нет, можно, конечно, поездить с беседами по школам. И, скорее всего, придётся, ибо план есть план. Но когда наблюдаемый говорит и движется - значительно легче. Опять же, близость солнцестояния...


 - Не нужно дёргаться, гражданка Берёзкина! Я всего лишь проверяю, не пронесли ли вы под платьем оружие! Стандартнейшая процедура! Таак... Здесь тоже нет. Садитесь, - полковник подвинул девушке стул, а сам сел в кожаное кресло. Кресло было слегка потёртым, но тут уж выбирать не приходилось. Городское УВД и так выделило лучший кабинет, который нашло.
Интуиция продолжала упорно сверлить горло. Где-то что-то есть. Где-то близко...
 - Дайте Вашу сумочку.
Девушка с неприязненной беспрекословностью протягивает. Может быть, "поиск оружия" был и лишним, но, чёрт возьми, имеет полковник право позволить себе небольшое удовольствие после тяжёлого рабочего дня? Да и вообще, день ещё не закончился...
В сумочке, разумеется, личного дневника не обнаружилось. Но обострённая интуиция говорила силовику: раз девушка пишет стихи, то он есть. И там всё-таки есть кое-что очень интересное... Ладно. Значит, по методичке.
 - Расскажите о своих увлечениях.
 - Учусь в основном... Стихи вот, видите, пишу... Больше писать не буду! Люблю читать...
 - У Вас есть домашние животные?..
Здесь можно расслабиться, задавая любые вопросы, и особо не слушать ответы. Главное, чтобы допрашиваемая не заметила главного...
 - Что Вам обычно снится?
 - Ой, ерунда всякая. Вот недавно снилось, что Пушкин и Есенин из-за меня стреляются на стенах Кремля... - девушка осеклась, испугавшись, что последние слова были лишними.
 - Занятно, - мужчина демонстрирует дружелюбие. Страх надо дозировать, чёётко дозировать... - А кто-нибудь из знакомых не снится?
 - Снятся, конечно, - Катя всё никак не прекратит тревожно тереть руки друг о друга, сжимая их коленками.
 - И какие же самые запоминающиеся сны со знакомыми?
 - Честно говоря, не знаю...
 - Эротические? Не стесняйтесь, говорите...
 - Ну... В некоторой мере. В общем, на олимпиаде по русскому я познакомилась с одним мальчиком, он на класс старше... И на следующую ночь мне снилось, как он ко мне пристаёт...
 - И что же тут необычного?
 - Понимаете, он через пару дней написал мне по интернету... В общем, он сказал, что снился мне специально, и пересказал сон. Нет, Вы не подумайте, у нас с ним ничего не было, я ему сразу сказала, что он не в моём вкусе, и больше мы не общались...
Многолетняя выдержка легко скрыла, что на этих слова у полковника сердце-таки ёкнуло. Переписка - раскрытая книга, парня он выцепит легко. Но теперь надо отвлечь её, отвлечь...
 - А какие парни в Вашем вкусе?
 - Знаете, вот как Фет в молодости...
Потерзав девушку подобной болтовнёй ещё с полчасика для приличия, безопасник наконец резюмировал:
 - Я вижу, что в Вашем произведении не было злого умысла. Вы, должно быть, просто написали его в спешке, не обдумав как следует?
--- Понимаете, меня вызвали к директору и сказали, что на покупку сценария денег нет,
 и нужно, чтобы я написала, а если я не могу написать сценарий, то пятёрки по литературе не заслуживаю, а я...

--- Успокойтесь. Такие, кхм, пида... гоги не стоят того, чтобы Вы переживали.
Пишите стихи, пишите, просто осторожно и вдумчиво. У Вас это получается.

"А директора ждёт сюрприз, - подумал он про себя, - мне тут всё равно крутиться, справки наводить. А коллегам раскрываемость. И девушки отблагодарим..."
 - Вы можете идти, - сказал он девушке, по-прежнему испуганно сидевшей на стуле, и слегка приподнялся с кресла, пожимая ей руку. - Всё будет хорошо, медаль с Вас никто не снимет.


