


--- Вот такой базар, Кирпич. А дальше сам думай. На тебе свет клином не сошёлся.

Бандит потёр складки кожи на затылке и снова уставился на девушку, лицо которой скрывала лёгкая чёрная вуаль.

--- А не боишься?

--- Кто понял жизнь, тот не спешит. Кто понял гит, тот не боится, --- неожиданно резко отчеканила девушка.

<<Сектантка? --- подумал Кирпич, --- ну и хрен с ней, золото-то, похоже, настоящее...>>

--- Можешь показать его знакомому ювелиру. Я уже сказала, что никуда не спешу, и потому срок --- неделя.
Ровно через семь суток ты кидаешь мне на мыло координаты закладки с калашами где-нибудь в лесу.
Если не выгорит --- цацку оставишь себе, но хрен тебе, а не следующий заказ. Уразумел? --- сказала ему Катрин.

--- Уразумел, краля. А ты всё равно поосторожнее...

--- Да я сама инструкция по технике безопасности! Кстати, --- скромная белая флэшка едва слышно стукнула по столу,
--- вот тут есть парочка оччень интересных документальных видео. В хайрезе, не хрен собачий.
Так что если со мной что случится --- топтать тебе зону, как пить дать, топтать...

\emptypar

--- Ну чё, Кипарис? --- авторитет Кирпич напряжённого оторвал спину от кожаного кресла, --- чё нарыл?

--- Цацка не подставная. Пробирка не стоит, но хрена ли мы хотим? Два-Брильянта-Три-Карата заберёт и оближется.

Под толстым черепом Кирпича шёл мыслительный процесс. Десять АК и пара сотен патронов за килограммовый слиток ---
это просто охренеть как выгодно. Такого навара просто не бывает, где-то подстава... Или просто кому-то позарез нужны стволы.
Для охраны шахты, ясен пень, для чего ж ещё. Кто ж, интересно, так подсуетился?
И какого лешего не крышу просят, а сразу оружие? Бойцы есть, а стрелять не из чего? Непорядок...

--- А что датчики? --- спросил он подчинённого, --- или ты лоханулся и нихрена ей на одежду не навесил?

--- Довели до леса, дальше борода. Как сквозь землю провалилась.

<<Ну точно, шахта, --- подумал Кирпич, --- надо лес прочёсывать>>.

\emptypar

За время, прошедшее с последнего универского медосмотра,
Катрин уже начала было отвыкать от стояния голышом под пристальным взглядом.

--- Теперь всё четко --- опуская растопыренную ладонь, произнесла Линдариэль.
--- А где-то в одежде посторонний металл. Извини, но надо было убедиться, что это не застёжка от лифчика.

--- Да ничего, --- потянулась Катрин за трусами, --- а можешь точнее определить, где?

--- Сейчас переберём, --- хищно улыбнулась эльфийка, --- и найдём. Ты же её распотрошишь, верно?

Программистка улыбнулась в ответ:

--- Распотрошу, непременно распотрошу. Я всё-таки не гентушница,
чтобы из-за каждого умника застёжку от лифчика туда-сюда отпарывать. Они у меня ещё попляшут...

\emptypar

--- Значит, так. Кипарис и Брухавый заходят слева, Чека и Вантуз --- справа.
Бульдозериста в расход не пускать, тупо на мушку. Малым ходом продвигаться вдоль свалки. Искать выход из шахты...
Кипарис, если твоя побрякушка наврала --- лично тебя здесь закопаю.

Детина, такой же бритый, как и его начальник, вздохнул. А что делать,
если за неделю был принят всего один сигнал длительностью меньше минуты?

\emptypar

Хранительница усмехнулась:

--- О, юные авантюристки... Но если так хочется...

...Портал-дециметровка соединил карантинный бункер с городской свалкой.
 Брезгливо поморщившись, Линдариэль сунула туда свой посох и запустила <<Поцелуй пустоты>>.
 Заклинание в силу отсутствия опыта вышло так себе, но в небоевых условиях эльфийка могла позволить себе его повторить.
 Когда воронка поглощённого мусора достигла в диаметре метров десяти
 (по крайней мере, через узкое окошко и сладковатый смрад подругам показалось именно так),
 Катрин сунула в пространственную дыру металлический прут,
 на конце которого был торжественно водружён разоблачённый датчик передвижения.
 Эльфийка продолжала лупить посохом куда-то вниз и вбок.

--- Ладно, хватит, --- сказала наконец студентка, --- задохнёмся же на фиг.

--- Хватит так хватит, --- пожала плечами её компаньонка, вынимая посох.
Хранительница, наблюдавшая за процессом через зеркало, тотчас захлопнула портал.

\emptypar


--- Нихрена себе, --- присвистнул Кирпич с кресла, спешно установленного на кабине бульдозера его шайкой и поправил ремень.

Открывшееся зрелище впечатляло.
Ровно в том месте, координаты которого сообщило устройство, красовалась здоровенная неровная воронка.
И что самое интересное --- оплавленных краёв, обугленного пластика или каких-то других следов взрыва видно не было.
Кирпич был простоват и туговат, но полные дураки главарями не становятся.
<<Втянули мусор в шахту>>, --- подумал он. Воображение уже живо рисовало ему подпольный комбинат,
легко и непринуждённо добывающий золото буквально из грязи. Видимо, в электронике действительно есть драгметаллы, и немало...
И всё это у него под носом! За какие-то крохи! Нет уж, так дело не пойдёт... Он их из-под земли достанет! В прямом смысле!!!

--- Кипарис, охренеть! А теперь копайте!

Робкие возражения бульдозериста на тему того, что до земли тридцать метров отходов и вообще это не земля,
а бетонная площадка, были пресечены стволом под нос и лопатой в руки.

\emptypar

<<Быть мэром, конечно, хорошо, --- думал Совиный-Козодоев, небрежно потягивая дорогое кофе в своём кабинете,
--- но вот дыра в бюджете... Может, я всё-таки переусердствовал и пятый этаж на вилле был лишним?>>

--- Артемий Тимурович, к Вам журналистка, --- пропел голос секретарши, --- хочет взять интервью по поводу бюджета.

Кабинет сразу показался мэру каким-то неютным. Докопались...

--- Никого не пускай, Галенька, --- сказал он в трубку, --- и через пять минут зайди ко мне, стресс снять надо.

В конце концов, для чего он эту блондинку держит?

Допить кофе и предаться любовным утехам нерадивый чиновник не успел.
В кабинет, игнорируя решительное <<К нему нельзя!>> длинноногой Галеньки, уверенным шагом вошла девушка на каблуках,
и вуаль на её шляпе заколыхалась в такт с планками жалюзи.

--- Что вам здесь надо? --- плавно повышая голос, взревел Совиный-Козодоев, не убирая ног со стола,
--- Проваливайте в свою редакцию!

По лицу Катрин, скрытому вуалью, пробежала невидимая улыбка.

--- А я не оттуда. Но тоже по поводу бюджета. Как мне стало известно из электронных архивов, значительные объёмы средств...

--- И какое ваше дело? --- перебил её мэр.

--- Предлагаю сделку. Вы мне --- оргтехнику вот по этому списку,
--- Катрин щелчком пальцев отправила пару листиков, скрепленных степлером, через стол,
--- а я убираю Южную свалку почти до основания.

Чиновник скептически усмехнулся:

--- Вам бы не ко мне, а к психиатру. Авось в одну палату с Наполеоном попадёте...

--- Я серьёзно, --- голос девушки ничуть не изменился, --- и я готова выполнить свою часть сделки первой.

--- Тогда приходите через месяц, обсудим план работ на ближайшую пятилетку, --- ощерился чиновник.

Катрин ничего не ответила, отвесила насмешливый поклон и вышла прочь.

\emptypar

--- Кир, да не знал я ничего!

--- Темнишь, Сова, ой темнишь... Позавчера, значит, здоровенная такая куча мусора была, а сегодня её уже нет?
И что за барышню описывает твоя тёлка?

--- Сумасшедшая она! Сумасшедшая!

--- А на хрена ты её на работе держишь, если она сумасшедшая?

--- Да не Галка! Девка эта! Сказала, что всё ликвидирует, а мне останется только тендер подписать! --- оправдывался мэр.

Кирпич вздохнул и отпустил его галстук.

--- Дай-ка угадаю, просила стволы и расходники к ним?

--- Нет. Три принтера, сканеры, мелочь всякую типа флэшек.

Список до сих пор лежал у Совиного-Козодоева в ящике, но доставать он его благоразумно не стал.

Кирпич поскрёб затылок. Вот, значит, как... Для чёрной бухгалтерии, что ли?

\emptypar


