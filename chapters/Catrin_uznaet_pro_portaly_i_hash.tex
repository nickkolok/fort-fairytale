...И сойдутся железные искры в медные, медные же в серебряные, серебряные в золотые, и разобьются на снопы. И загорятся звёзды Огня, Земли, Воды и Воздуха, и одна промеж ними внутренняя, и разлетятся они, но вспыхнут лучи и восстанут узы. И лягут элементы пара против пары, и рванутся от внутренней, но не вырвутся, а сомнут ткань пространства.
Катрин глубоко вздохнула, покачала головой, легким движением пальцев поправила невидимый изъян на причёске.
- Я, конечно, дико извиняюсь, а нормальной документации нет?

- Это канонический текст, описывающий открытие портала, дитя моё, - с еле слышной укоризной проговорила Хранительница, - он недостаточно понятен для тебя?
- Можно сказать и так. Я привыкла иметь дело с более простыми объектами... Вы не могли бы объяснить на пальцах?

Хранительница пожала плечами:
- Мне отец рассказывал... Портал открывается заклинанием, что, вообще говоря, является редкостью в магии Форта, которая большей частью образна... Заклинание слагается из букв - кроме J, Q, W, X, Y. Услышав его из уст Хранителя, Сердце порождает те самые искры... Искры как-то взаимодействуют между собой; именно описание способа их взаимодействия и повергло тебя в такое удивление. И при этом, к слову говоря, выделяется довольно много энергии... Если после того, как искры отсверкали, останется ровно пять звёзд, которые слагаются из высших искр, золотых, и ещё низшие искры, железные - то портал возникнет. Если нет - просто выброс энергии зря...

... Сейчас мне известно два таких заклинания. Одно - то, которым мы открываем спасательный портал. Другое же, увы, бесполезно, поскольку результат имеет вдвое большие размеры... И эти два заклинания были упомянуты тут, в этом манускрипте, который ты столь пылко нарекла недостойным светлого звания "документации". Занятные, однако, у вас слова... Ещё до Исхода наши мудрецы пытались силой расчёта предсказать, какие ещё сочетания букв открывают пространственный проход - но тщетно. И решено было: каждый день Хранитель, вознадеясь на интуицию свою, предпринимает попытку. И вносит результат в хроники... О, я вижу, ты хочешь спросить, почему лишь одну, дитя моё? Отвечаю: расточительно было бы предпринимать больше.
- Да нет, я не это хотела спросить, - опустила голову Катрин, - я хотела попросить черновики расчётов. Уж что-то, а хэш-функцию сбрутить - как два байта переслать, был бы алгоритм...
- Неужто ты хочешь сказать, Катрин, что ваша цивилизация знает способы поиска заклинаний, да к тому же связывает их с именем Цезаря? - магичка вскинула брови.
- Да нет, заклинания тут не при чем. Но руками брутить хэш... Вы б ещё биткоины на бумажках майнили!
- Дитя моё, в речи твоей мне знакомы разве что предлоги; хоть и полагаю, что объяснение всех новых для меня явлений будет сложным и объёмным, а заглядывать в память в поисках столь абстрактных понятий остерегаюсь, природное любопытство не даёт мне покоя... Скажи мне главное: я верно тебя поняла, что за сотню лет наука внешнего мира продвинулась в достаточной мере, чтобы иметь способ нахождения нужного сочетания букв? Значит ли это, что скоро нам грозит вторжение?
- Да, продвинулась. Нет, не грозит, порталы на момент моего бегства изобретены не были. По крайней мере, я ничего об этом не знаю. Обещать ничего не могу, более того, поколдовать над запуском ноутбука придётся... В прямом смысле. Ноутбук - это такой переносной компьютер, который...
- Не утруждайся покамест, дитя моё. Я попрошу у смотрителя библиотеки черновики тех расчётов... Тебе же предстоит в ближайшее время знакомить свой артефакт с нашей магией, я верно разумею?
- Верно, верно, - заулыбалась беглянка, и в глазах её заплясали озорные искорки. - Поживём - увидим. Чем рандом не шутит?
