Девушка силилась расцепить пальцы, сжимающие руль, но не могла этого сделать около минуты.
Всё тело трясло мелкой дрожью, левая рука дёргалась на ключе зажигания.
Наконец она коснулась спиной кресла. Получилось. Выдачи отсюда нет. Только бы дверь не заклинило...
Ничего, ласточка, мы с тобой ещё поездим... Можно ведь и без глушителя, правда?
Задние колёса, похоже, во рву остались, ну да ничего...

<<Я тебя с тремя пересоберу. Запаска-то цела. Вот увидишь...>>

Дверь, к счастью, не заклинило, и беглянка, сумев осмотреться по сторонам, дёрнула ручку.
Вокруг машины уже стояли четверо стражников --- не то конвой, не то почётный эскорт.
Пока девушка приходила в себя, Хранительница успела спуститься с башни
и уже спокойно шагала по серой брусчатке площади Приветствия.

--- Добро пожаловать. --- Голос хозяйки звучит спокойно и уверенно,
хотя глаза выдают некоторое волнение. Или это так кажется?

Девушка аккуратно прикрыла дверцу. Не захлопнулась... Но закрывать с размаху неуместно.
Подумав чуть, она приоткрыла машину так, чтобы было видно, что дверь не защёлкнута. Мало ли, куда её повезут!

--- С прибытием! --- повторила Хранительница.

--- Здрав-вствуйте.

Беглянка не знала, стоит ли делать книксен или ещё какое ку, но хозяйка не дала неловкому молчанию повиснуть:
сделав ещё шаг навстречу девушке, он протянула руку.
Чуть вверх нарочито распрямлённой ладонью, чтобы было сразу понятно, что это не для поцелуя.

--- Мари дель Кастелладо.

Беглянка прикоснулась к протянутой руке --- и растерялась окончательно. Почему-то она не думала о встрече;
о том, как добраться --- думала; о том, чем займётся --- по крайней мере пыталась. А вот о знакомстве...

--- Не бойтесь, дитя моё. Вы устали. Просто представьтесь --- любым именем, для внешнего мира Вы умерли.
Можете перечислить все, можете придумать новое...

--- Жужелицына Екатерина Степановна... Можно просто Катя... И на <<ты>>, Вы же старше...
Ой, извините, я имела в виду, по положению... по званию...

--- Не переживай, меня сложно обидеть словом или жестом. Как и любого из нас.

--- А ещё... --- Мысль, острая, как игла шприца, пронзила Катино тело от головы до ног и обратно вверх, до грудной клетки
 и она заговорила быстрее, --- на форумах... мой ник Катрин2101... Что с Зайкой?!

--- Та девушка, что прибыла сюда несколько недель назад? С ней всё хорошо, не беспокойся. Так вы с ней знакомы?

--- По форуму... По переписке... А...?

--- Тоже всё в порядке. Маги едва успели --- но успели. Повидаться вы сможете едва ли ранее, чем дней через восемьдесят.

\emptypar

Видя, что девушка облегчённо выдохнула, Хранительница продолжала:

--- Пожалуй, из всего, что ты перечислила, привычнее всего для нас будет Катрин. Катрин де...?

--- А почему вдруг?..

--- Твоя первичная эгидаль --- Белая Лилия. Дворянская французская школа... Должен быть определённый род.

--- Катрин де Лада, --- подал голос один из стражников, указывая на металлический значок на <<Копейке>>.
--- Тут и герб имеется...

Девушка испуганно замолкла, уже успев представить, как её отсюда выкинут... Или вкинут в подвалы.
Хранительница же снова проявила проницательность:

--- Если ты стыдишься рода или изгнанница... Не бойся. Не ты первая, не ты последняя.
Не беда, что твой герб незатейлив.
Мы берём отнюдь не по роду.
А по Силе. Если бы не Сила...

Женщина сделала приглашающий жест и повернулась к башне, Катрин последовала за ней.

--- ...то ритуал вызова бы просто не сработал. В него ведь надо что-то влить...
А Сила, дитя моё, передаётся по наследству.
Ты спросишь меня, действительно ли мир несправедлив настолько, что одни дети от рождения превосходят других?
Не настолько уж Сила отлична в этом отношении от богатства. Даже родившись без неё, можно её получить.
Есть отдельные практики... Есть стэки, генераторы, концентраторы... Не буду тебя утомлять, почти пришли.

--- А можно ещё вопрос?

--- Вопрос задать ты можешь всегда, а получить ответ...

--- У Зайки какая Сила?

--- Зайка --- как ты её называешь, она была почти без сознания, представиться, конечно, не успела --- исключение.
Когда мы нашли её... Думаю, ты понимаешь, что отказать не могли. Разовьём.
Научим. Стэки свободные на Сердце висят, это всё суета.

Они поднимались по винтовой лестнице, однако, отнюдь не крутой; после третьего витка Катрин уже было начала сомневаться,
не уснёт ли она на ходу, но тут её спутница свернула на этаж и, небрежно, с каким-то привычным достоинством, вскинула руку;
ладонь плавно, но быстро распрямилась, следом, после секундой паузы, напряглись пальцы.
Тяжёлые дубовые двери аккуратно раскрылись, открывая вход в небольшой зал.
Камин, драконья шкура на полу, несколько клинков на стене, да стол, накрытый скатертью.

--- Кто ж нам, двум беззащитным дамам, двери откроет? --- улыбнулась Хранительница после того,
как закрыла их похожим жестом, только в обратном порядке, и в глазах её блеснули искорки. ---
Прошу к столу, такова традиция. Обычно к нам прибывали узники инквизиции... В казематах не очень-то сытно кормят.

/*Слот под историческую справку*/

--- Спасибо... Я тоже... Давно нормально не ела.

Конечно. Солёные огурцы из подвала да яблоки. Что ещё добудешь, когда банковская карточка заблокирована?..

\emptypar


\emptypar
