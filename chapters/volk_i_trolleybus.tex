

 - Берта, Рекс, взять!
Полковник ОПОНа изволит развлекаться на дежурстве. Весь лично-хвостатый состав, освобождённый от намордников, загоняет бездомного пса и... и средства, списываемые на корм собакам, идут на пиво руководству. Офицеры стоят в сторонке, периодически прикладываясь к прохладным металлическим баночкам. Здоровенная серая псина - почти настоящий волк, как такое могло из бездомных собак наплодиться? - отбивается как-то странно. Короткими укусами, будто нехотя, и норовит ускользнуть.
 - Фас! Ату его!


...Ты едешь в самом конце салона старенького троллейбуса, неподвижно уставившись в ободранную листовку "Все на выборы-2048!", закрывающую тот кусочек окна, что свободен от рекламы, и комкаешь пробитый билет на одну поездку. Впрочем, "едешь" - это громко сказано. Так, медленно плетёшься по набережной родного Севастополя вместе с остальной пробкой...


Очередная овчарка прыгает, целясь жертве в горло. И снова промах. Атака, промах. И опять. Наконец серый, кажется, начинает уставать. Движения его меняют характер... ну же, ещё одна атака... Берта падает с перегрызенной шеей. Рекс кидается отомстить - и остаётся без лапы. Полковник, как и положено начальству, выходит из замешательства первым:
 - Огонь!
Три ствола щёлкают почти одновременно, три куска разозлённого свинца попадают в цель - и не причиняют ей вреда. Волк перепрыгивает поджавшего хвост Мухтара и вцепляется в запястье ближайшему ОПОНовцу. Пистолет летит в снег, откуда немедленно перекочёвывает в зубастую пасть. Недавняя жертва лихо перемахивает двухметровый забор, едва коснувшись колючей проволоки, и растворяется в почти застывшем потоке транспорта.


...Сигнальщик завершает передачу. Флажки замирают. Минута - и дракононосый корабль тает в тумане...

...Ты знаешь, что почти наверняка троллейбус остановили из-за тебя. Что ты не будешь "выходить по одному" вместе со всеми, а останешься здесь, чтобы накинуть к цене своей жизни пару лишних минут рабочего времени полиции. И, дождавшись, пока бабулька аккуратно сложит газету с ребусами, поправит очки и неторопливо доковыляет до выхода, вспрыгиваешь на сиденье.

...Гибралтар. Сардиния по левому борту. Крит. Босфор. Дарнанеллы. Бриг летит быстрее пинга, при таких-то задержках на СОРМах...

Прыжок. Ещё прыжок. Вот когда не надо - она едет, и прыгай тут, сбивая когти об крыши этих железных повозок. Задержали, сволочи, успеть бы... хорошо хоть пули не серебряные. А вот и наш товарищ...

...Ты знаешь, что будешь отбиваться. Как сможешь. Не только и не столько для того, чтобы выместить свою ненависть. А чтобы тебя застрелили. С этим сейчас просто. И ты хочешь избавить себя от будущего искушения рассказать об ошибке в расчётах Национального Стелларатора в обмен на улучшение условий заключения. Переход к пределу не в том пространстве, какая пошлость...

...Предупредительный выстрел патрульного катера вывел корабль из молчания.
 - Casus belli! - грянуло над палубой. - Огонь!
Книппель прочерчивает воздух, срезая флаг Конфедерации с адмиралтейства.

Молоденький сержант под одобрительные смешки коллеги врывается в троллейбус с криком "Полиция! Всем лежать!" и бежит прямо к забившемуся в угол парню, на ходу выхватывая одной рукой - пистолет, другой - удостоверение. Преступник, напряжённо сжав зубы, готовится, очевидно, оказать сопротивление при задержании...

Немезида оторвалась от бушприта и заработала крыльями, покрывая "мёртвый кабельтов"...

...Прыжок. Последний. Серая туша насквозь пробила стекло и скатилась с сиденья. Волк встал на дыбы... и по проходу неслышно зашагал, поднимая трофейный "Макаров", юнец в серой мантии со свежими прорехами, собирая лучи вечно конвульсирующих "газосветок" в звёздочки на плечах и медальон с башенками на шее.
 - Контринквизиция. Оружие на пол.



