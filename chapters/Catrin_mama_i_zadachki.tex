%Сомнительно. Утверждается, что неправдоподобно.

--- Мам, а придумай мне ещё задачку!

-- Катенька, солнышко, - усталая женщина гладит школьницу по пшеничным волосам, --- я же давно училась в школе...
Я не помню уже химию! --- и обреченно прикрывает глаза. Надо вспомнить. Как бы ни утомила работа, надо.
Ведь это всё из-за неё. Не смогла. Не уберегла. Не заработала.
Ведь её сероглазое чудо, её последняя надежда, умница и отличница не виновата,
что у её родителей нет денег на оплату дополнительных задач.
Она не виновата, что переписывание условия в тетрадку обложили непомерным налогом,
оставив бесплатными только по пять-шесть задач на каждую тему.
Она не виновата, что её одноклассники могут, но не хотят покупать себе упражнения. Не виновата...

--- Три моля хлорида кальция помещают в раствор, содержащий 105 грамм едкого натра...

И в том, что квартира напичкана камерами, Катя тоже не виновата. Квартира ведь в ипотеку.
А не взять кредит --- значит, не соблюсти нормальные условия содержания ребёнка.
Придут ювеналы... и так, мол, отец погиб, да ещё и мать содержать нормально не может.
И всё, прощай, семья, здравствуй, детдом. И стервятники-психологи. Нет, отдать дочь она им не могла.
Так и учились --- в основном летом, в старом деревянном доме прадеда,
до которого ещё надо было добраться от автобусной остановки.
Зато в нём были книги. И можно было говорить о чём угодно...

<<Ничего>>, --- думала она, --- <<Катя обязательно выучится.
Вон она какая умница. Исполнится восемнадцать - водить научится.
Будем на дедовой "копейке" ездить, будет меня в деревню возить. Она умница.
С компьютером с этим вон как ловко обходится, хоть и старый он. Прорвёмся...>>
