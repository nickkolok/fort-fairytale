- Если бы всё было так просто, дитя моё! - Хранительница покачала головой и грустно рассмеялась.
Девушка уставилась в стол, разглаживая скатерть:
- И что же мешает?
- Расстояние. Думаешь, до тебя никто не хотел вернуться и водрузить знамя справедливости в том понимании, которое имел? У многих наших предков ведь всё отняли. Положение в обществе, честное имя, дом, у кого-то и семью... А Форт даёт очень много. Защиту от чумы. От многих болезней. Силы. Даже регенерацию... пусть и долгую и болезненную. Умение понимать любой язык и быть понятым любым человеком. Да Форт и не отбирает свободы, просто... Всё упирается в недостаток магии. Для бытовых и учебных нужд генераторов хватает с лихвой; порталы же даются ценой значительных трат. А чтобы поддерживать все данные возможности для человека, однажды ступившего на землю Форта, нужна магия. И если человек удалится от Форта вдвое, то расход её возрастёт вчетверо. В первой же беседе с тобой я упоминала, что большинство магов ведут упорную работу именно над экономией...
Хранительница замолкла. Катрин поправила волосы, задвинув выбившуюся прядь.
- А как же вылазки за пленниками?
- Это считанные минуты, дитя моё. Разрезать путы, схватить страдальца - и на мост. А вот внедрить своих людей в общество внешнего мира - уже не получится. Мы и пленных не берём поэтому, кстати.
Серые глаза Катрин медленно разгорались. Бывшая студентка свела руки под подбородком и опёрла на них голову. Где-то там, в глубине зрачков, с грохотом неслись поезда мыслей, перекидываемые, словно костяшки, по рельсам невидимых счётов. Когда молчание начало затягиваться и Хранительница уже готова была начать рассказывать очередную то ли легенду, то ли быль - а знающий о реальности Форта вряд ли рискнул бы проводить между ними чёткую границу - Катрин вдруг негромко и чётко сказала:
- Спасибо. Теперь у меня есть цель.


Через несколько секунд буря во взгляде успокоилась, и Катрин почувствовала себя обычной ученицей - примерно так, как бывало на семинарах, когда стареющий доцент Громостонко рассказывал об очередной книге. Книге, естественно, недоступной для обычного студента и даже преподавателя - но не для Катрин. Она твёрдо знала, что, придя домой, запустит верный AfterTOR Browser - и немедленно вопьётся в сочную мякоть абстракций.
- Извините... Я хотела бы начать адаптацию со знакомства с магией. Хоть котлы драить, хоть посох подавать. Главное, чтобы спрашивать можно было. И чтобы отвечали.
- Я поняла тебя, дитя моё; посох, говоря мимоходом, чужому в руки ни один маг не даст, да и зелья у нас не распространены... Я честно признаюсь, что испытываю радость и облегчение от того, что ты сама пришла к этому решению; я опасалась, что подходящего дела для тебя не найдётся и твой вклад в совершенствование Форта ограничится лишь беседами, как было с Фоссетом.
Он положил всю свою жизнь на то, чтобы вычислить наше местоположение и убедиться в реальности Затерянного Города Z. Он убедился - и потерял все желания. Путь назад ему был заказан, да и не очень манил. Так и довольствовался мой дед прогулками с ним да разговорами.
Что же касательно твоей просьбы о наставнике - я не могу дать ей удовлетворение прямо сейчас. Видишь ли, отроки в твои годы уже имеют солидный запас знаний об окружающем их магическом мире, и потому даже как ассистент ты будешь совершенно бесполезна. Вместо этого я постараюсь подобрать тебе компаньонку... именно девушку, чтобы это не выглядело навязыванием тебе пары во имя процветания разнообразия крови... с которой вы будете проводить время вместе.
- Спасибо, - Катрин серьёзно кивнула, - я постараюсь оправдать ожидания... и Ваши, и свои.
