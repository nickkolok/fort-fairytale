Министр тяжело вздохнул. Взглянул на дверь бункера, на стол, прошёлся взглядом по лицам собравшихся людей,
остановив его прицел на командующем РВСН.

--- Товарищ генерал, вы разобрались с этими вашими "Сумерками"?

--- Никак нет, товарищ министр безопасности, --- голос звучит чётко и ровно, ибо офицер не должен трусить,
--- выяснить ничего нового не удалось. Со штатного портативного передатчика,
закреплённого за радистом экипажа пропавшего комплекса, было получено сообщение морзянкой:
"Подобрали вампирку, хотели тра...". Передача оборвалась двумя точками, что соответствует либо законченной букве "и",
либо незаконченным "ж", "с", "у", "ф", "х", "э" или "ю". Я всё же полагаю, что там было "х"...

--- Полагать будет комиссия. Является ли код Морзе штатным средством передачи информации?

--- Никак нет. Инструкция предписывает использовать его только в случае, когда факт передачи необходимо скрыть.
Передатчик расположен в одежде так, что процесс отправки выглядит как нервное подёргивание...

--- Ваша версия произошедшего?

--- Я склонен верить радисту, - ничего лучшего просто не удалось придумать, - подобный бред трудно выдумать.
 Факт же приёма данного сообщения неоспорим. Вы знаете, \_как\_ это регистриуется.

--- То есть, по-вашему, экипаж ракетного комплекса в тайге наткнулся на вампирку и решил выместить на ней своё сексуальное влечение?

--- Не исключено. Физические нагрузки у ракетчиков на марше невысоки, в основном они сидят в подвижных установках, так что желание...

--- А "Дрын-М" и "Грабли-13Ё" тоже вампирка унесла на чёрных крыльях? Предварительно закусив телами вместе с одеждой и оружием? А потом обильно помочилась с воздуха на всё это безобразие?

--- Не могу знать, товарищ министр.

--- Всё с вами понятно. Что может сказать руководитель экспертно-криминалистической группы?
Конечно, со всеми давно переговорено с глазу на глаз. Почти достоверно известно, что не врёт ни один. Но вдруг вместе удастся обнаружить противоречия, а то и выявить предателя? Или хотя бы до чего-то лельного дойти.

--- Практически ничего, - очкарик в белой форме полковника научных войск пожимает сутулыми плечами, - следы крови и фрагменты тканей организма есть. Анализ ДНК показал, что с вероятностью семь сигма кровь принадлежит некоторым членам экипажа. Также есть следы крови животных, вполне типичных для данного региона и едва ли представляющие интерес. Мышь, куропатка. Предположительно смерть животных наступила в результате естественного пищевого процесса хищников. Точнее сказать сложно из-за уже упомянутых следов воды. Гидродинамисты говорят, что поток падал практически отвесно, скорость - примерно 20 м/с, диаметр струи - 5-8 дециметров. Всего было сброшено порядка тысячи тонн... Практически полностью сброшенная вода слилась в близлежащий овраг.

--- Конечно, следов никаких?

--- Разумеется. На хвое и так их почти невозможно различить, а после такого размытия...

--- Есть ли необычные обстоятельства, которые могут привлечь внимание? Товарищ полковник, постарайтесь припомнить.
Важна каждая деталь...

Полковник, который, честно говоря, предпочитал, когда его называли профессором, тяжело вздохнул.

--- Аномально низкое содержание дейтерия и, предположительно, трития в пробах оставшейся воды.

--- О чём это говорит?

--- Ни о чём. При стандартном осмотре места происшествия это не замеряется. Но в этот раз, с учётом "ядерного" контекста,
мы решили проверить изотопный состав воды. Там практически нет ни дейтерия, ни трития, ни тяжёлого кислорода.

Кто бы мог подумать, что эта строчка в инструкции когда-нибудь пригодится!

--- То есть боеголовка частично сдетонировала?

--- Скорее наоборот. Нейтронов в воде меньше, чем должно было бы быть. Процесс, способный на месте изымать нейтроны из вещества и направлять их, например, в ракету с целью её деактивации, я даже представить не могу. Скорее, подобная особенность характерна именно для сброшенной воды.

--- Я правильно помню, товарищ полковник, - министр напрягся, нависая над столом, - дейтерий может использоваться в термоядерном синтезе?

--- Не только. Сброшена была, если можно так выразиться, обеднённая вода, из который весь дейтерий извлечён. Существующие методы добычи дейтерия весьма разнятся, но я не знаю ни одного, который бы понижал его содержание в отходах настолько.

--- То есть вы полагаете, что противник сбросил отходы?

--- От "облегчённой" воды вреда нет.

Министр поправил часы и посмотрел на четвёртого человека за столом.

--- Разведка, вам слово.

--- 8 августа, то есть примерно на неделю позже похищения ядерной ракеты, на американской базе FX-4201 патруль обстрелял НЛО. В перехваченном документе говорится о том, что НЛО приняли за беспилотник из-за небольших размеров. Ракета TR-563 не произвела на НЛО видимого действия.
9 августа было перехвачено сообщение об аварийном сходе с орбиты двух спутников военной радиосвязи США. Результат работы следственной группы перехватить не удалось... но мы работем над этим.
11 августа поступило сообщение о нападении на объект FX-1468, более известный как "Нью-Нокс". Источник нападения не определён, оружие, применённое нападавшими, не было радиоактивным, выживших нет, часть трупов найдена. Параллельно эти неизвестные каким-то образом устранили беспилотники и подавили радиосвязь. Экономические последствия, полагаю, общеизвестны.
12 августа - приказ отказаться от обстрела любых летательных аппаратов. Естественно, мы им воспользовались.
Ходят слухи о некой металлической пластине, найденной чуть ли не на лужайке Белого дома, но пока больше похоже на дезинформацию.
Собственно, у меня всё.


--- Благодарю, коллеги.

Картинка потихоньку складывалась.

--- Если предположить, что эти два инцидента связаны --- очень уж короток временной интервал ---
то ситуация получается весьма тревожной. Сначала некто похищает новейший образец нашего воружения.
Настолько секретный, что он перемещался по малообитаемым местам по маршруту, известному только непосредственному командованию.
Оно допрошено?

--- Так точно, товарищ министр, - отрапортовал генерал РВСН, - на полиграфе. Ничего не знают.

Ага, конечно. Ладно-ладно, пусть думают, что главным там был полиграф, а не молоденький паренёк из конвоя.

--- Отсюда первый пункт --- противник сумел вычислить местоположение комплекса.
А если сложить дейтерий, НЛО и неизвестное оружие?

--- Инопланетяне? - скептически ухмыльнулся разведчик, --- не верю.

--- А кто? Китай? Положим, вычислить координаты они смогли бы. Что это за "вампирка" --- оставим пока за скобками.
Может, вообще не человек, а резиновая баба. Или мы совсем неправильно поняли сообщение. Но куда делась техника, куда?

--- До ближайшей реки шесть километров. Зона размытия к ней не примыкает. Остались бы следы.

--- А под землю?

--- Исключено. Всё прочёсано металлоискателями и сейсмографами. Подземный ход также исключён ---
маршрут формируется на краткий срок, за который выкопать нужный тоннель невозможно.

--- Воздух?

--- Никак нет. Радары ничего не засекли над этим участком тайги.

--- Вот-вот, --- оживился профессор, --- антигравитационные платформы как раз подходят!
Погрузили на них, на небольшой высоте, метр-два, провели по тайге куда надо...
Собственно, не обязательна даже антигравитация, достаточно любого метода полёта, не оставляющего следов.
Да с чего вы вообще взяли, что это противник?

--- А кто, по-вашему?

--- Например, мирная исследовательская экспедиция более разумной расы, отправленная в наш "заповедник"...

--- А если эти мирные исследователи в своих мирных исследовательских целях посадят в кабину "Дрына" какого-нибудь психопата?

Учёный замолк.

--- А если действительно вампиры? - заговорил ракетчик, - мы ведь не знаем, сколько может унести один вампир.
А их там могло быть несколько.

--- Начнём с того, что, не считая легенд, это первое свидетельство их существования.
Даже если допустить, что они унесли ракету и станцию, откуда вода?

--- Да кто вообще сказал, что ракету куда-то унесли? --- снова влез профессор, --- Может, они её того...
в воду аннигилировали. В виде принудительного разоружения. И спутники за тем же посбивали.

--- А Нью-Нокс?

--- Мало ли! Вдруг там разрабатывалось оружие, которое им сильно не понравилось?

Разведчик кивнул, запоминая версию для проверки. Министр покачал головой:

--- А при чём тут вампирка?

--- Голограмма, например, для отвода глаз. Или результат галлюцинаций у отдельно взятого радиста.
Следов известных наркотиков не обнаружено, но это ещё не значит, что не был применён неизвестный галлюциноген,
например, электромагнитной природы.

--- Давайте доработаем версию с земным противником, --- сказал министр, --- итак, требования к нему:
бесследная транспортировка в условиях тайги, использование галлюциногенов (или вовсе костюмированного представления,
что тривиально), создание беспилотных летательных аппаратов, орбитальное оружие, нерадиоактивное оружие большой мощности.
Второе и третье подходит много к кому, орбитальное оружие - не исключаю, что у Китая или Южного конуса есть,
но в принципе можно и обычной ракетой. Чем громили Нью-Нокс и как вывезли "Дрын"?

--- Базу теоретически можно каким-нибудь лазером с орбиты. И им же спутники, --- ответил учёный.

--- Это было бы прекрасно, но как он там оказался?

--- Китай запускал спутники. Южане тоже. Кто проверял, что там на самом деле?

--- Допустим. Прошу разведку проработать... А вода и "Дрын"?

--- Людей перебили теми же лазерами с беспилотников. На них же везли воду. После чего воду сбросили, технику подцепили...

--- Прям все в одной точке воду сбросили?

--- Хорошо. Одна платформа. Сбросила воду, подцепила технику, попутно перебив военных.
Или не так: сначала подцепила технику, потом поняла, что перегруз, сбросила воду. На сверхмалых высотах, лесом --- до границы.

--- То есть как минимум два аппарата? Маленькая "тарелка" и большая "платформа"?
И как тогда объяснить невредимость "тарелки" после обстрела?

--- Сбили ракету на подходе. Тем же "лазером". Существующие источники энергии настолько компактными не бывают,
но если у них термоядерный реактор на основе воды... Всё равно не верю, - вздохнул учёный,
--- тут токамак-то в плюс не выведешь. Если бы были научные разработки - где-нибудь бы что-нибудь опубликовали.

--- Разведке - проверить, - скомандовал министр и продолжил, - а теперь посмотрим, что можно смело перенести на инопланетян.

--- Да всё можно! --- заявил профессор, --- термояд у них запросто может быть,
спутники на орбите --- сколько угодно, неядерное оружие --- тоже...

--- В общем, --- подытожил министр, --- куда копать разведке --- понятно.
РВСН --- прекратить курсирование ракетных комплексов. Остановить, где сейчас стоят, и окапываться.
Заодно и физнагрузка будет, чтоб не "тра..." всё, что движется.
Экспертам... перепроверить ещё раз изотопный состав всего, включая деревья. Совещание окончено.
