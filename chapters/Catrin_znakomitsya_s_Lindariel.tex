

- Да нет, Катрин, я так не умею - звонко рассмеялась девушка, - и спала ты сегодня сама. Так что я просто подождала.
Беглянка, приподнявшись, рассматривала гостью, небрежно сидевшую на спинке её кровати.

В первую очередь, конечно, в глаза бросались заострённые ушки, плавно выбивавшиеся из-вод распущенных рыжих волос. Немного пухловатые, маленькие губки эльфики чуть разошлись в стороны в едва заметной улыбке; зелёные глаза с неестественно большими радужками любопытно смотрели из-под длинной чёлки.
 - Так, значит, ты та самая компаньонка, которую мне выделили?
 - Угу, - рыжая соскочила на пол, - так что давай что-нибудь делать!
Катрин наконец сунула ноги в ярко-голубые кроссовки - те самые, которые выжимали сцепление перед мостом, - и окончательно вырвалась из уютного плена кровати. Примерно минуту девушки смотрели друг на друга. Катрин никогда не видела такой странной обуви - как будто босоножки на платформе, но почему-то серой и блестящей, почти без разницы по высоте между пяткой и носком. В такой обуви компаньонка была чуть выше её; но более всего удивила гостью одежда эльфийки, состоявшая из короткой помеси юбки с шортами и неприлично открытого топика-лифа; материал того и другого по фактуре напоминал тёмную кожу и, как заметила Катрин, великолепно контрастировал с белоснежной талией самой эльфийки.
 - Упс, - заморгав глазами, сказала вдруг компаньонка, - кажется, я забыла представиться. Извини. Линдариэль.
 - Катрин, - заулыбалась беглянка, пожимая протянутую ладонь - бледную, с тонкими пальцами, - ты уже знаешь.
 - Эм... Так чем займёмся? Сама понимаешь, я первый раз!


 - Давай для начала приведём меня в порядок, - улыбнувшись, рассудила беглянка, - во-первых, нужно перенести мои вещи из автомобиля и отодвинуть его с прохода, во-вторых, покажи мне, пожалуйста, как тут стирают одежду.
Эльфийка запрокинула голову, прищурив правый глаз:
 - Что такое автомобиль - мне рассказала Хранительница. А что такое "стирать"?
Катрин удивлённо уставилась на собеседницу:
 - А вы что, вещи просто выбрасываете и наколдовываете новые?
 - Да зачем из выбрасывать? - продолжила Линдариэль эстафету непонимания.
Катрин выдохнула. И начала терпеливо, как при нелегальном репетиторстве за продукты объясняла ученикам теорему Виета, рассказывать:
 - Сначала одежда новая. В ходе ношения она пачкается. От пота, от пыли и так далее. Носить такую одежду становится неприятно, она воняет и изменяет цвет. Тогда её стирают...
 - Так. Стоп. С какого перепугу она пачкается?
 - Линдариэль, либо я чего-то не понимаю... - Катрин растерялась, - Одежда - это пористый материал. Ткань или трикотаж. При контакте с жидким или сыпучим веществом происходит проникновение частиц вещества между волокнами одежды...
 - Ой банки-склянки! - эльфийка сделала рукой нечто средне между фейспалмом и хлопком по лбу, - она у тебя не заговорённая что ли?

 - Наконец-то! Ну сама подумай,
