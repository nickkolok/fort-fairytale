...зимою волк, сердитый волк,

Трусцою пробегал... \upcite{v_lesu_rodilas_elochka}

Влад недовольно дотянулся нижней губой до того места, где начали прорезаться усы.
Слёт одарённых детей --- это, конечно, хорошо и интересно. Экскурсии в Сколково, форумы там всякие...
И что характерно --- даже ни одной кабальной бумаги подписывать не нужно!
Не то что для получения стипендии --- там надо согласиться до конца жизни работать где скажут.
Но с этим дурацким утренником --- явно перебор!

...Лошадка мохноногая

Торопится, бежит...

Они же, в конце концов, отличники, олимпиадники. Серьёзные люди. С шестого по одиннадцатый класс.
Неужели организаторы не понимают этого настолько,
что приглашают водить хороводы под руководством этого жалкого подобия Деда Мороза?
Не могли даже полушубок на него одеть... Так, красный пиджачок и брюки.
Вылитый Санта-Клаус, тот самый, которого собираются запрещать. Или уже?

...Срубил он нашу ёлочку...

А впрочем, ради экскурсии на Национальный Стелларатор можно и подпеть.
Если бы ещё кто-нибудь рассказал, почему он не работает...
Но говорят, что из-за каких-то принципиально новых квантовых эффектов, которые отечественная наука вот-вот откроет.

...И много-много радости

Детишкам принесла!

<<Недодед>> выдержал паузу, сунул руку куда-то в глубины пиджака... и серо-будничным, отнюдь не сказочным, голосом сообщил:
--- В соответствии с поправками к закону
<<О трудовой повинности, возникающей при нарушении права на интеллектуальную собственность>>,
вступившими в силу с 1 января сего года...

Дальше Влад не слушал, обречённо глядя на то, как у двери встают двое ОПОНовцев.
Цепочка сплелась, мозаика сложилась. Конечно, так и бумажка не нужна. А были ли среди приглашённых уже <<крепостные>>?
Или только те, кого заполучить хотелось, но не получалось?

Додумать он не успел. Дверь снова раскрылась, и в зал вошли ещё двое.
Опираясь на посох, Дед Мороз в синем узорчатом полушубке затянул извечное:

--- Здравствуйте, детишки, девчонки и мальчишки...

--- Это ещё что такое? --- недовольно шагнул вперёд <<Клаус>>-провокатор, снова доставая удостоверение.

--- А вот это мы должны спросить, что это такое, --- сверкнул из-под кокошника колючий серый взгляд Снегурочки.

--- По-моему, внученька, вон тот Дед Мороз --- ненастоящий! --- прозвучало из-под скрывавшей лицо белой бороды.

ОПОН пока не вмешивался. Даже <<калаши>> с предохранителя не снимал. Мало ли, как там оно задумано...

--- Такой же ненастоящий, --- ухмыльнулся провокатор, --- как и вы, гражданин... Как вы сюда вообще попали?

--- Почему это я ненастоящий? --- проигнорировал второй вопрос <<синий>>, --- Настоящий! У меня даже посох волшебный!

<<Клаус>> только скептически ухмыльнулся.

--- Раз, два, три... --- завёл седобородый...

--- Арестовать их, --- кратко и спокойно скомандовал провокатор. Охрана сделала шаг, готовясь заломить девушке руки...

--- ...ёлочка, гори!

\emptypar

Посох мгновенно взлетел и замер параллельно полу.
Тугая, хлёсткая струя вязкого пламени ударила прямо в лицо <<Клаусу>>, плотоядно расползаясь по одежде.
Тонкая длинная серебристая шпага, соткавшись из воздуха в руках Снегурочки, описала полукруг, рассекая шею ближайшего ОПОНовца.

%TODO: В помещении явно дым. Вписать открытие окна или что-то в этом духе.

--- За гитхаб и битбакет! \upcite{resh_o_zaprete_git_2027}

К тому времени, как девушка расправилась со вторым, её напарник уже протягивал ей трофейный <<Калашников>>.

Так Катрин стала первой в истории ученицей Громостонко, получившей от него автомат.

Вбежавший на крики боец был успокоен короткой очередью в упор, возродив позабытую традицию:
последним словом полицая было <<партизаны>>.
Ещё на троих, дежуривших под окнами, Громостонко спустил <<спартанского лиса>>, и пока мужчина лутовал трупы,
Катрин вышла в центр зала, обращаясь к не успевшим ничего понять школьникам:

--- Копирасты и угнетатели хотели вам показать, \textbf{что} они могут с вами сделать.
Мы показали, что можем сделать \textbf{мы}. Слава Пиратскому Партизанскому Интернационалу!

Новогодняя парочка скрылась в коридоре.
Через две минуты камера в туалете мигнула и погасла навсегда, так что никто не мог видеть,
как стена с писсуарами превратилась в двухметровый портал.


Заслушав отчёт об экспедиции, Хранительница одобрительно кивнула:
--- А детей точно не расстреляют?
--- Не расстреляют, --- качнула ресницами Катрин, --- я прямую трансляцию на телек пустила.
Зря мы, что ли, два микропортала открывали под кабель?

