Омск дремал.
Январь вступал в свои права, неслышно проходя по дворам и проспектам.
Никто не претендовал на его владения: не ревели моторы бульдозеров,
не лязгали лопаты дворников, даже вороны, казалось, разучились каркать.
Январь неторопливо, со степ\'eнным удовольствием осматривал нетронутый снег.

Корпоратив приближался к той фазе, когда тамаду будить уже некому, а салаты давно стали братской подушкой.
Охранник скучал на своём посту; игра про то, как отважные российские ракетчики сбивают американских свиней ---
Angry Bears --- давно обрыдла хуже пареной редьки. Охраннику было, в общем-то, всё равно, кто там празднует.
Да и что празднует --- тоже. Контора --- и контора. При мэрии --- стало быть, при мэрии.
Его дело маленькое --- никого не пущать.

Секьюрити вдохнул и снова выдохнул. Делать было решительно нечего...

...Январь, заглянув на очередную улицу, задохнулся от возмущения.
По девственно-белому снегу, по \textbf{его} снегу, беспардонно поднимая белую пыль, неслись сани.
Тринадцать песцов резво тянули упряжь; пассажиров было двое.

Присмотревшись внимательнее, Январь немного успокоился:
незнакомцев вполне можно было счесть некоей вариацией на тему Деда Мороза и Снегурочки ---
посох и кокошник были на подобающих местах --- только почему-то одетых в чёрное.

Охранник поднял голову. Закрыл глаза и снова открыл.
Дед Мороз спрыгнул с саней, галантно подавая руку спутнице.

 --- Вы вообще кто? --- спросил удивлённый секьюрити.

 --- А мы вот как раз сюда, --- ответил дед.

 --- Зал забронирован на сутки... --- начал исполнять охранник свои охранительные обязанности..

 --- Послушайте, милейший, --- бесцеремонно перебил его гость, --- идите отсюда и не мешайте. Целее будете.

Охранник, оживившись, потянулся за висящей на поясе резиновой дубинкой.
Как бы то ни было --- со скукой было покончено.

Дед взмахнул посохом вроде бы и не сильно, но секьюрити, пролетев метра три,
с горестным матом приземлился на копчик в ближайший сугроб.
Мужчина с полупоклоном открыл дверь девушке, и оба скользнули внутрь.

 --- Минуточку внимания! --- наставительно подняла палец гостья.

 --- В-вы ещ-щё кто та-такие? --- с трудом подняв мутный взгляд, спросил один из праздновавших.

 --- К хорошим программистам на Новый Год приходят Дед Мороз и Снегурочка, а к плохим --- мы,
 Дед Лайн и Несмогурочка!

 Катрин развязала пояс. Края чёрной накидки разошлись в стороны, и в её руки скользнул автомат.

  --- Принимайте подарки! --- и девушка вдавила гашетку, чувствуя плечом сладкую тяжесть отдачи.

\emptypar



Министр медленно отложил отчёт. Потом резким движением рванул его к себе, пробежался по паре строк, снова отложил.
Вдохнул. Выдохнул. Подвинул бумагу к себе, проводя пальцами по краю листа, словно проверяя на реальность.

Лист проверку выдержал.
Министру безопасности оставалось лишь включить планшет и наглядно увидеть теракт.
Первый теракт за двадцать лет, о котором министр узнал после совершения.






