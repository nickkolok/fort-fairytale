--- Смотри, как красив закат. Как летают над волнами чайки... Смотри и будь безмятежна. Всё будет хорошо, я рядом.
Как же здорово, что ты дождалась... Вот видишь, сошёл на берег, а без моря не могу.
Что значит <<дорого>>? Я премию получил. И вообще, ты --- самое дорогое, что у меня есть.
Как я мог не поделиться с тобой солнцем, медленно тающим в открытом море... Я очень хочу, чтобы ты запомнила наш медовый месяц...

Девушка



\emptypar

\emptypar

/*

Внезапно налетает буря. Прогулочная лодку с парусом получает серьёзные повреждения, затем раскалывается пополам.
Девушку начинает относить от парня-моряка. Моряк видит <<Летучий Голландец>>

*/

--- Слушай сюда, будь ты хоть сам морской чёрт! Я готов, дьявол тебя побери, отдать свою жизнь в вечное служение тебе,
только пусть она останется жива! Слышишь, призрак? Слышишь?!

Голос его потонул в шторме. Корабль исчез --- и тут же появился снова, почти рядом с парнем.
Моряку осталось только протянуть руку к канату, свисающему с борта.

\emptypar

Сигнал бедствия резанул Астрал, и, не успела отгреметь троекратное <<Все на стены!>>,
как тут же вспыхнул сигнал тау-один.

/*

<<Летучий голландец>> уходит с места кораблекрушения за несколько секунд до того, как Форт открывает портал.
Провести присягу невозможно --- и девушку из воды вытаскивают на драконе.
Через несколько часов, обсушив её магией, под присягу Форту её всё-таки подводят.

\emptypar

Катрин, пользуясь давней дырой в базе МЧС (а вернее --- объединённого Министерства Безопасности),
начинает выуживать сводки о кораблях, подающих сигнал бедствия, и отслеживать их зеркалами.
На очередном эпизоде обнаруживается <<Голландец>>.
Хранительница принимает решение открывать портал --- через зеркало много не поймёшь.

Со стен Форта адепты замечают семафорную передачу, которую Катрин пытается расшифровать.

*/



Катрин ворочалась. Второй день загадка семафоров не давал ей покоя.
Она уже наизусть выучила и русский, и международный вариант семафорной азбуки --- послание не складывалось.
Кадры из наспех заснятого видео стояли перед глазами, но упорно отказывались сложиться в цельную картинку.
Даже если учесть, что начало послания принять не успели.
И что корабль-призрак непредсказуемо прыгнул в портал, не дав завершить передачу. Не может быть такая мешанина из букв. Не может.

Девушка пихнула кулаком ни в чём не повинную подушку. Они же всё проверили! И русский вариант.
И международную семафорную азбуку. И в каждом случае --- зеркальное отражение сигналов. Даже гипотезу о контрамоции рассмотрели...
Тщетно. Упрямые закорючки никак не желали складываться в сколь-либо осмысленный текст. Значит, шифр.
Почему? Зачем потребовалось кодировать послание? Для кого? Василий ведь не знает, что такое Форт.
Или сигналили не им, а шифр был заранее условлен? И какого Вейерштрасса на него не работает зеркало?
Он вообще живой человек или <<ходячий мертвец>>?
И если последнее, то, получается, Форт столкнулся с очень сильной некромантией?
И вбиваемая ученикам магов с младых ногтей истина <<у мёртвого энерготоков нет>> --- неверна? Брр. Не на ночь!
А чтобы не зацикливаться на том, что защита Форта может и не помочь от потусторонних сил, надо подумать о шифре.

Значит, предположим, Василий успел как-то с кем-то договориться.
Предположим далее, что этот сигнал был адресован тому судну, которое было на грани крушения. Нет, не годится!
<<Голландец>> перемещается произвольно. Или нет?
Но передача оборвалась совершенно точно внезапно, чуть ли не на полузнаке, паузы не было. Ладно, фиг с ним...
Здоровый девичий сон всё же брал своё, образы становились мутными, растекаясь по внутренней стороне век... Выспаться...
Завтра напишем программку... Может быть, простой шифр со сдвигом. Это самое простое... А может быть, шифр-замена.
В универах этому не учат, но мы-то знаем, как анализировать частоту...
Мы-то всю эту цензуру не раз и не два нагинали и граблями, граблями, с поворотом на пи пополам...
Значит, программка... И функция подсчёта... Граблей... Не, букв. Значит, функцию мы назовём howFrequentInMessage...
Нет, плохо... compareFrequenceWithStandart... Тьфу, там же не Standart, а Standard... Это по-русски Т, а по-английски D, и...

Катрин рывком вскочила. Сердце учащённо билось. Программистка включила осветительный шар и, наспех обув балетки,
побежала в лабораторию Школы Янтарной Мудрости.

\emptypar

--- Итак, --- девушка, устало, но довольно улыбаясь, подбросила пальцем прядь, --- после расшифровки имеем:
<<... дер Декен, капитан и последний член экипажа, умираю от старости и голода.
Корабль не заражён ни одной известной болезнью...>>.

--- Как? --- удивлённо захлопала заспанными глазами Линдариэль.

--- Транслит. Международная семафорная азбука исключена из курса молодого бойца <<в целях поднятия патриотизма>>
приказом минбезопасности... не помню какой номер. Видимо, наш товарищ счёл,
что самое важное содержится в последней записи судового журнала, и попытался её передать.
Как сумел. Язык --- очевидно, голландский. Из-за отсутствия пробелов мы его в кириллице не разглядели...
Я транслитерировала и подобрала пробелы. Экспериментально, чтобы получались слова. Вот, собственно и всё.

Линдариэль изогнула брови:

--- И из-за этого правда стоило меня будить?

--- Стоило. Более того, я намерена разбудить Хранительницу. Если капитан действительно пишет, что остальные умерли,
то никаких скелетов и призраков нет. Судя по описанию очевидцев, команда <<Летучего Голландца>> вполне могла писать письма;
значит, будь у них посмертие, в журнале бы это отметили.
А значит, бояться некромагии нечего, более того, корабль бы неплохо исследовать...

--- Так. Слушай меня. Сейчас ты ложишься и спишь, --- строго сказала эльфийка, --- ложишься. И спишь. Поняла?

--- Поняла, --- кивнула девушка. Честно говоря, она и сама была не против.

--- ...а завтра ещё раз всё обдумываешь и докладываешь. Такие вещи ночами не делаются!


%/*
%дердекенкапитейненхетлаастелидвандебеманнингомтестервенваноудердом...
%derdeckenkapiteinenhet laatste lid van de bemanning om te sterven van ouderdom en dood. Het schip is niet besmet is door een bekende ziekte
%*/

\emptypar


\emptypar

